%===============================================================
% \chapter*{Discussion}\addcontentsline{toc}{chapter}{Discussion}\markboth{Discussion}{Discussion}
%===============================================================

%===============================================================
\chapter*{Results}\addcontentsline{toc}{chapter}{Results}\markboth{Results}{Results}
%===============================================================

This thesis set out to address a common but---at our faculty---underexplored issue: the abandonment of student-created games after their academic purpose has been fulfilled. Our goal was to design a practical and accessible support system that could help students further develop and potentially commercialize their creative projects.

We successfully launched a recruitment website and collected over a dozen contacts from students interested in the initiative. Association rules were created, and a testing mechanism was prepared to validate the model. While no legal entity was established due to time constraints, we consider this stage of the project a success in laying the conceptual and operational groundwork for future implementation.

Our goals were largely met. We researched the necessary background, analyzed comparable support structures, described a relatively concrete mechanism, and~tested user sentiment. Along the way, we also gained new insight into how timing within the academic cycle and peer collaboration influence entrepreneurial outcomes and learned a surprising amount about the commercial effectiveness and usefulness of cooperatives.

However, several open questions remain. Can the initiative be sustained long-term? How should it be funded during times of financial instability? Will we be able to achieve continuity across academic years? Will the executive mechanisms work? These are important challenges that this project must address.

For further development, we recommend refining the internal processes based on feedback, gradually introducing more formal structure and responsibility and altering the rules of association if necessary. Collaboration with faculty leadership and external partners could strengthen the model and broaden its~impact.

This thesis does not offer a final solution but rather a starting point---a working concept that, with continued effort and iteration, can empower students to~transform their academic projects into meaningful, real-world ventures. We wish our successors the best of luck.

%===============================================================
% \chapter*{Conclusion}\addcontentsline{toc}{chapter}{Conclusion}\markboth{Conclusion}{Conclusion}
%===============================================================
