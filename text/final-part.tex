%===============================================================
\chapter*{Results and Discussion}\addcontentsline{toc}{chapter}{Results \& Discussion}\markboth{Results \& Discussion}{Results \& Discussion}
%===============================================================
In the research I encountered a wide range of institutional support systems for game developers, including broad innovation incubators, pre-incubators, and dedicated game incubators. While these models offer valuable services—such as mentorship, workspace, networking, and business development—they typically operate with a broader entrepreneurial context in mind. Their target audience often consists of aspiring founders at an earlier stage in their journey, their processes are less specialized, and sometimes involve equity stakes or~longer application cycles.

In contrast, the student-made projects at FIT CTU tend to be further along in development, often already boasting functional prototypes, strong technical execution, or unique design ideas. They, however, lack the entrepreneurial direction needed to turn them into commercial products. Importantly, these students are not necessarily aspiring entrepreneurs at the outset—they are developers first.

As such, the support system I have designed takes a different approach. It~is intentionally lightweight, progresses through the formalities quickly, and is open to all students regardless of their prior business interest or experience. Unlike incubators that may aim for equity ownership and financial returns, my model focuses on incorporation and scale. The goal is to remove friction, lower logistical barriers, and propone entrepreneurship.

Because of the narrow scope of expected projects and a well-defined audience, I have a clearer understanding of the risks associated with incorporating these ventures. This insight allows me to believe this model is feasible and capable of making a meaningful impact.

%===============================================================
\chapter*{Conclusion}\addcontentsline{toc}{chapter}{Conclusion}\markboth{Conclusion}{Conclusion}
%===============================================================
In this thesis, I set out to address a common but—at our faculty—underexplored issue: the abandonment of student-created games once their academic purpose has been fulfilled. My goal was to assess the current situation, examine mechanisms for commercialisation, and design a programme—its contents, steps, and future vision. I aimed to validate this approach and describe a pilot run.

I studied the broader ecosystem of academic ventures, the game development process both generally and within our faculty, and the types of systems that support these projects. I also investigated the strengths of cooperative models and the complexities of co-owned intellectual property, which are especially relevant in student-led collaborations.

I designed a mechanism, outlined its steps, and provided an outlook for its~future. I concluded that the administrative burden of running a business is the primary obstacle students face and proposed a legal framework to address it—drafting statutes and providing a setup guide. I launched a recruitment website, promoted it through a campaign, and began collecting contacts from students interested in developing their games and those eager to support the~initiative. I also described a test run to validate the solution. I consider this stage a success in laying the conceptual and operational groundwork for further implementation.

However, responses collected so far via the website are insufficient to guide improvements based on user sentiment. The legal entity must still be formally established, staffed, and teams enrolled before the test run can proceed and the mechanism assessed.

Several questions remain open: Will the initiative gather traction, attract student interest, and staff? Will the programme be enjoyable and helpful? Can the initiative be funded and sustained long-term? Will the legal framework work? Will students and staff be willing to governan it? These are important points that Grafit.games will have to answer, address, and potentially overcome.

This thesis does not offer a final solution but rather a starting point---a working concept that, with continued effort and iteration, can empower students to~transform their academic projects into meaningful, real-world ventures.
