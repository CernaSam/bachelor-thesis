%===============================================================
% \chapter*{Discussion}\addcontentsline{toc}{chapter}{Discussion}\markboth{Discussion}{Discussion}
%===============================================================

%===============================================================
\chapter*{Results and Discussion}\addcontentsline{toc}{chapter}{Results \& Discussion}\markboth{Results \& Discussion}{Results \& Discussion}
%===============================================================
In the research I encountered a wide range of institutional support systems for game developers, including broad innovation incubators, pre-incubators, and dedicated game incubators. While these models offer valuable services—such as mentorship, workspace, networking, and business development—they typically operate with a broader entrepreneurial context in mind. Their target audience often consists of aspiring founders at an earlier stage in their journey, their processes are less specialized, and sometimes involve equity stakes or longer application cycles.

In contrast, the student-made projects at FIT CTU tend to be further along in development, often already boasting functional prototypes, strong technical execution, or unique design ideas. They, however, lack the entrepreneurial direction needed to turn them into commercial products. Importantly, these students are not necessarily aspiring entrepreneurs at the outset—they are developers first.

As such, the support system I have designed takes a different approach. It is intentionally lightweight, progresses through the formalities quickly, and is open to all students regardless of their prior business interest or experience. Unlike incubators that may aim for equity ownership and financial returns, my model focuses on incorporation and scale. The goal is to remove friction, lower logistical barriers, and propone entrepreneurship.

Because of the narrow scope of expected projects and a well-defined audience, I have a clearer understanding of the risks associated with incorporating these ventures. This insight allows me to believe this model is feasible and capable of making a meaningful impact.

%===============================================================
\chapter*{Conclusion}\addcontentsline{toc}{chapter}{Conclusion}\markboth{Conclusion}{Conclusion}
%===============================================================
In this thesis, I set out to address a common but—at our faculty—underexplored issue: the abandonment of student-created games after their academic purpose has been fulfilled. My goal was to assess the current situation, mechanisms used for commercialization and design a programme—its contents, steps, and vision for the future. I aimed to validate my approach and describe a pilot run.

I have studied the broader ecosystem of academic ventures, the game development process both generally and within our faculty, and the types of systems that support these projects. I also investigated the strengths of cooperative models and the complexities of co-owned intellectual property, which are especially relevant in student-led collaborations.

I have designed a mechanism, its steps and outlook for the future. I have arrived at the conclusion that the administrative burden of running a business is the main obstacle in a student’s path and proposed a legal framework to address this issue—created statutes and provided a setup guide. I have successfully launched a recruitment website, promoted it in a campaign, and started collecting contacts from both students interested in developing their games as well as those willing to further the initiative. I have also described a test run that will be used to validate the solution. I consider this stage of the project a success in laying the conceptual and operational groundwork for further implementation.

Nevertheless, replies so far collected through the website are insufficient in guiding modification based on user sentiment. The legal entity needs to be formally established, staffed and teams enlisted before the mentioned test run can be carried out and the mechanism assessed.

Several questions remain open too. Will the initiative gather traction and attract student interest? Will it be successfully staffed? Will the process be successful in preparing games for launch? Will it be enjoyable? Can the initiative be funded and sustained long-term? Will it endure times of financial instability? Will we be able to achieve continuity across academic years? Will the legal framework work? Will students and staff be willing to participate in the governance and administration of a large entity? These are important points that Grafit.games will have to answer, address, and potentially overcome.

This thesis does not offer a final solution but rather a starting point - a working concept that, with continued effort and iteration, can empower students to transform their academic projects into meaningful, real-world ventures.
