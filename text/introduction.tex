%===============================================================
\chapter*{Introduction}\addcontentsline{toc}{chapter}{Introduction}\markboth{Introduction}{Introduction}
%===============================================================
Each year, students pour creativity, effort, and technical skill into developing original games as part of competitions and their studies. Yet—despite their potential to grow into successful products—these projects are often left unused once they are handed in. Discover with me why that happens and what can be done to address this issue.

The thesis focuses on student-created games at FIT CTU in Prague. While the local environment shows promising signs—especially through strong private sector investment—many student projects fail to progress beyond the classroom. I believe this is largely due to a lack of entrepreneurial confidence and know-how among students, as well as the complexity and effort involved in commercializing a game. In this work, I aim to investigate how a simple, accessible support system could help bridge that gap.

I will assess support currently available for game developers at different stages of the creative process and the possible mechanisms for commercialization. My goal is to design a programme tailored to my fellow students’ needs—its steps, what the mechanism should provide and which directions it should strive to expand in the future. To validate my approach, I will conduct user sentiment testing and then describe a test run—including how its success or failure will be evaluated—and plan subsequent steps. Ultimately, my broader objective is to improve entrepreneurial skills and mindset within the local community and to inspire and motivate students to launch their own game ventures, equipping them with practical tools and confidence to navigate the industry.

Rather than going deep into the technical aspect of game development and directing of a university-incubator, the work will instead focus on preparing the conceptual foundation for future implementation. This approach will allow me to test interest and refine the model before any formal financial commitments are made. Explore with me how student creativity can be supported and transformed into real-world success.
