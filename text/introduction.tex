%===============================================================
\chapter*{Introduction}\addcontentsline{toc}{chapter}{Introduction}\markboth{Introduction}{Introduction}
%===============================================================
Each year, students pour creativity, effort, and technical skill into developing original games as part of their studies. Yet once these projects are handed in and graded, they are often left unused---despite their potential to grow into successful products. This thesis explores why that happens and what can be done to change it.

Our focus is on student-created games at FIT CTU in Prague, Czechia. While the local environment shows promising signs---especially through strong private sector investment---many student projects fail to progress beyond the classroom. We believe this is largely due to a lack of entrepreneurial confidence and know-how among students, as well as the complexity and effort involved in commercializing a game. This thesis aims to investigate how a~simple, accessible support system could help bridge that gap.

This thesis examines the support currently available for game developers at~different stages of the creative process. Our goal is to consider the possible mechanisms for commercialization, evaluate and propose a system tailored to~our students’ needs. To validate our approach, we will conduct user sentiment testing and then describe a test run, including how we will evaluate its success or failure and plan subsequent steps.

Rather than going deep into the technical development of games, the selection and financing required to sustainably run an incubator, or creating a functioning university spin-out company, this thesis will instead focus on preparing the conceptual foundation for future implementation. Our goal is to design a~system---its processes, rules, and intended outcomes---to prepare the project for implementation. This approach allows us to test interest, gather feedback, and refine the model before any formal or financial commitments are made.

The thesis is divided into sections---each made up of several multiple chapters. The research part explores the game development process, academic entrepreneurship, and commercialization models---both locally at FIT CTU and internationally. In the design part we compare options and select a model suited for our needs. Then we move to practical steps: we prepare association rules, launch a recruitment website, collect interest and prepare a validation mechanism. The thesis concludes by reviewing the results, assessing whether the goals were met, and outlining future steps. We invite you to explore with~us how student creativity can be supported and transformed into real-world success.