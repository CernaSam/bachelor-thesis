%===============================================================
\chapter{STANOVY DRUŽSTVA}\label{ap:statutes}
%===============================================================
\textbf{\Large Grafit.games, výrobní družstvo}

%===============================================================
\paragraph{I. Základní ustanovení}
\begin{enumerate}
    \item Obchodní firma družstva: Grafit.games, výrobní družstvo.
    \item Sídlo družstva: Praha.
    \item Předmětem podnikání družstva je:
        \begin{enumerate}[label=\alph*.]
            \item vývoj, výroba, distribuce a servis počítačových, mobilních a konzolových her, včetně všech souvisejících činností (např. grafika, hudba, scénář, marketing, QA, vydavatelská činnost, pořádání herních akcí, vývoj nástrojů, konzultace, školení).
            \item obchodní, marketingové, reklamní a mediální služby v oblasti digitální zábavy.
            \item výzkum a vývoj v oblasti informačních technologií a digitálních médií.
            \item další podnikatelské činnosti související s vývojem a provozem herních projektů.
        \end{enumerate}
    \item Družstvo je společenstvím neuzavřeného počtu osob založeným za účelem podnikání. Družstvo je obchodní korporací podle zákona o obchodních korporacích.
    \item Výše základního členského vkladu, kterým se každý člen podílí na základním kapitálu družstva, činí 500,- Kč. Člen se za podmínek podle těchto stanov může podílet na základním kapitálu družstva současně i jedním nebo více dalšími členskými vklady. Členský vklad je tvořen součtem základního členského vkladu a všech dalších členských vkladů.
    \item Členem družstva se může stát jen fyzická osoba starší 18 let, tým fyzických osob starších 18 let nebo právnická osoba.
    \item Podmínkou vzniku členství v družstvu je splnění vkladové povinnosti k základnímu členskému vkladu.
    \item Družstevní podíl představuje práva a povinnosti člena plynoucí z členství v družstvu a lze jej nabýt za podmínek podle zákona a těchto stanov přijetím za člena družstva, nebo převodem od dosavadního člena.
    \item Statutárním orgánem družstva je předseda družstva.
    \item Družstvo zřídilo ve svém sídle a ve svém oficiálním komunikačním kanálu informační desku. Družstvo zpřístupňuje členům prostřednictvím informační desky údaje týkající se jeho činnosti určené zákonem, těmito stanovami nebo rozhodnutím orgánu družstva. Včasné a řádné uvádění těchto údajů na informační desce je povinností předsedy družstva.
\end{enumerate}
%===============================================================
\paragraph{II. Členský vklad}
\begin{enumerate}
    \item Základní členský vklad může být rozhodnutím členské schůze poměrně zvýšen všem členům z vlastních zdrojů družstva za podmínek a v rozsahu podle zákona.
    \item Členská schůze může rozhodnout o zvýšení základního členského vkladu doplatky členů. Základní členský vklad lze zvýšit doplatky členů pouze jednou za 3 roky a nejvýše na trojnásobek stávající výše. Lhůta ke splnění povinnosti člena k doplatku určená rozhodnutím členské schůze nesmí být kratší než 90 dnů a nesmí přesáhnout 3 roky ode dne přijetí tohoto rozhodnutí.
    \item Základní členský vklad nelze za trvání členství vracet; to neplatí, jestliže postupem určeným zákonem došlo podle rozhodnutí členské schůze ke snížení základního členského vkladu. Část, o kterou byl základní členský vklad snížen, vrátí družstvo každému členu, jehož členství ke dni zápisu snížení základního členského vkladu do obchodního rejstříku trvalo, do 1 měsíce ode dne tohoto zápisu.
    \item Základní členský vklad lze podle rozhodnutí členské schůze snížit také za účelem úhrady ztráty. Základní členský vklad se každému členu družstva sníží ke dni zápisu snížení základního členského vkladu do obchodního rejstříku.
\end{enumerate}

%===============================================================
\paragraph{III. Členství v družstvu}
\begin{enumerate}
    \item Členství v družstvu vzniká jen při splnění všech podmínek stanovených zákonem a těmito stanovami přijetím za člena.
    \item Přijetím za člena vzniká členství v družstvu uchazeči o členství, který je fyzickou osobou, týmem fyzických osob, nebo právnickou osobou a podal přihlášku, dnem rozhodnutí předsedu družstva o přijetí za člena nebo pozdějším dnem uvedeným v tomto rozhodnutí ve shodě s přihláškou.
    \item Tým je tvořen minimálně dvěma fyzickými osobami, které se dohodly na společném členství v družstvu a na vnitřních pravidlech rozdělení práv a povinností mezi sebou.
    \item Tým je v družstvu zastoupen jedním statutárním zástupcem, který vykonává práva a povinnosti člena družstva za celý tým.
    \item Vstup jednotlivých fyzických osob do týmu a jejich vystoupení z týmu se řídí vnitřní dohodou členů týmu, kterou tým předloží družstvu.
    \item Přihláška uchazeče o členství i rozhodnutí o přijetí do družstva musí mít písemnou formu a musí kromě adresy určené členem pro doručování v oficiálním komunikačním kanálu obsahovat:
    \begin{enumerate}[label=\alph*.]
        \item obchodní firmu a sídlo uchazeče v případě právnické osoby,
        \item jméno týmu, jména a bydliště členů v případě týmu,
        \item jméno a bydliště v případě fyzické osoby.
    \end{enumerate}
    \item Družstvo vede seznam členů, do kterého se zapisují (dále jen „zapisované skutečnosti“)
    \begin{enumerate}[label=\alph*.]
        \item jméno, příjmení a bydliště (u fyzických osob včetně členů týmů) nebo jméno a sídlo člena (u právnických osob), jakož i adresa určená členem pro doručování v oficiálním komunikačním kanálu,
        \item den a způsob vzniku a zániku členství v družstvu,
        \item výše členského vkladu.
    \end{enumerate}
    \item Člen je povinen bez zbytečného odkladu oznámit a na žádost družstva doložit každou v seznamu členů zapisovanou skutečnost i její změnu. Družstvo provede v seznamu členů zápis zapisované skutečnosti bez zbytečného odkladu poté, kdy se o této skutečností dozví; to platí také o zápisu změny takové skutečnosti.
    \item Údaje zapsané v seznamu členů družstva může družstvo používat pouze pro své potřeby ve vztahu ke členům družstva. Za jiným účelem mohou být tyto údaje použity jen se souhlasem členů, kterých se týkají. Člen má právo do seznamu nahlížet a žádat bezplatné vydání potvrzení o svém členství a obsahu svého zápisu v seznamu členů.
    \item Zapisované skutečnosti týkající se osob, které přestaly být členy družstva, vede družstvo v oddělené části seznamu členů, která je přístupná pouze předsedovi družstva. Předseda družstva umožní nahlížet do této části seznamu členů jen bývalému členovi, jehož se zápis týká, a jeho právnímu nástupci.
    \item Údaje zapsané v seznamu členů lze v případech neuvedených v předchozích ustanoveních těchto stanov zpřístupnit jen za podmínek podle zákona.
\end{enumerate}

%===============================================================
\paragraph{IV. Práva a povinnosti člena}
\begin{enumerate}
    \item Člen má v souladu se zákonem a stanovami právo
    \begin{enumerate}[label=\alph*.]
        \item volit a být volen za předsedu družstva,
        \item účastnit se řízení a rozhodování v družstvu,
        \item podílet se na zisku družstva,
        \item podílet se na výhodách poskytovaných družstvem,
        \item na vypořádací podíl při zániku svého členství za trvání družstva,
        \item na podíl na likvidačním zůstatku při zrušení družstva s likvidací.
    \end{enumerate}
    \item Člen je povinen
    \begin{enumerate}[label=\alph*.]
        \item dodržovat stanovy a podmínky užívání služeb,
        \item dodržovat rozhodnutí orgánů družstva přijatá v souladu se zákonem a těmito stanovami.
    \end{enumerate}
        \item Likvidační zůstatek se rozdělí mezi členy družstva rovnoměrně.
\end{enumerate}

%===============================================================
\paragraph{V. Družstevní podíl}
\begin{enumerate}
    \item Každý člen má družstevní podíl stejný. Podíl představuje práva a povinnosti člena plynoucí z členství v družstvu.
    \item Převést družstevní podíl nelze.
\end{enumerate}

%===============================================================
\paragraph{VI. Podíl na zisku}
\begin{enumerate}
    \item Na konci kvartálu, pokud to umožňuje solventnost družstva, členská schůze hlasuje o rozdělení výdělku družstva na základě řádné nebo mimořádné účetní závěrky. Po odečtení části 5%, která je určena k tvorbě fondu družstva má člen právo na podíl výdělku družstva odpovídající výdělku z jeho činnosti k poslednímu dni účetního období, za které se zisk rozděluje.
    \item U člena, jehož členství trvalo jen po část tohoto účetního období, se podíl na zisku poměrně krátí.
    \item Výdělek se rozděluje formou dividend. 
    \item V případě týmu se dividendy platí přímo členům. Každý člen má nárok na stejný podíl z rozdělované částky, přičemž podíl členů, kteří do týmu vstoupili v průběhu účetního období, je poměrně krácen.
    \item Podíl člena na výdělku je splatný do 3 měsíců ode dne, kdy členská schůze o rozdělení rozhodla.
\end{enumerate}

%===============================================================
\paragraph{VII. Zánik členství}
\begin{enumerate}
    \item Členství v družstvu zaniká
    \begin{enumerate}[label=\alph*.]
        \item dohodou,
        \item vystoupením člena,
        \item vyloučením člena,
        \item smrtí člena,
        \item zánikem právnické osoby, která je členem družstva,
        \item jiným způsobem určeným zákonem.
    \end{enumerate}
    \item Vystoupením členství zaniká dnem, kdy člen družstvu doručil písemné oznámení o vystoupení. To neplatí, pokud člen do 30 dnů ode dne, kdy členská schůze přijala usnesení o změně stanov, pro kterou nehlasoval, doručí družstvu písemné oznámení o vystoupení, ve kterém uvede, že vystupuje z důvodu nesouhlasu se změnou stanov. Členství v takovém případě zaniká uplynutím kalendářního měsíce, v němž bylo oznámení o vystoupení družstvu doručeno, přičemž změna stanov není pro vystupujícího člena účinná a vztah mezi ním a družstvem se řídí dosavadními stanovami.
    \item Člen může být z družstva vyloučen, jestliže závažným způsobem nebo opakovaně porušil své členské povinnosti. Rozhodnutí o vyloučení musí předcházet písemná výstraha předsedy družstva, ledaže porušení členských povinností, které je důvodem k vyloučení, mělo následky, jež nelze odstranit. Ve výstraze musí být uvedeno, jaké povinnosti člen závažným způsobem nebo opakovaně porušil a v čem porušení těchto povinností spočívá, a to spolu s upozorněním na možnost vyloučení a s výzvou, aby člen s porušováním členských povinností přestal a aby následky porušení členských povinností odstranil. K tomu se členovi vždy poskytne přiměřená lhůta, nejméně však 30 dnů od doručení výzvy.
    \item O vyloučení nelze rozhodnout později než ve lhůtě 6 měsíců ode dne, kdy se družstvo dozvědělo o důvodu k vyloučení, nejpozději však ve lhůtě 1 roku ode dne, kdy důvod k vyloučení nastal.
    \item Rozhodnutí předsedu družstva o vyloučení musí být písemné. Rozhodnutí musí obsahovat také poučení o právech vylučovaného člena podle zákona o obchodních korporacích, zejména o právu podat námitky k členské schůzi ve lhůtě 30 dnů ode dne doručení rozhodnutí o vyloučení a právu podat soudu návrh na prohlášení rozhodnutí o vyloučení za neplatné ve lhůtě 3 měsíců ode dne doručení případného rozhodnutí členské schůze o zamítnutí námitek a potvrzení rozhodnutí o vyloučení.
    \item Orgán, který o vyloučení rozhodl, může rozhodnutí o vyloučení zrušit s písemným souhlasem vylučovaného člena. Jestliže vylučovaný člen souhlas do 1 měsíce ode dne, kdy mu bylo rozhodnutí o zrušení rozhodnutí o vyloučení doručeno, neudělí, k rozhodnutí, kterým bylo rozhodnutí o vyloučení zrušeno, se nepřihlíží. Souhlas vylučovaného člena se zrušením rozhodnutí o vyloučení se nevyžaduje, pokud vylučovaný člen o zrušení rozhodnutí o vyloučení již dříve písemně požádal.
    \item Rozhodnutí předsedu družstva o vyloučení člena, stejně jako rozhodnutí členské schůze o zamítnutí námitek a potvrzení rozhodnutí o vyloučení, anebo rozhodnutí, kterým bylo rozhodnutí o vyloučení zrušeno, se doručují doporučeným dopisem do vlastních rukou na adresu člena uvedenou v seznamu členů vedeného družstvem.
    \item Členství vylučovaného člena v družstvu zaniká marným uplynutím lhůty pro podání námitek proti rozhodnutí o vyloučení nebo dnem, kdy bylo vylučovanému členu doručeno rozhodnutí členské schůze o zamítnutí námitek a potvrzení rozhodnutí o vyloučení.
    \item Družstvo nemůže uplatnit vůči vyloučenému členovi žádná práva plynoucí ze zániku jeho členství do uplynutí lhůty pro podání návrhu soudu na prohlášení rozhodnutí o vyloučení za neplatné, a jestliže byl takový návrh vyloučeným členem soudu podán, až do doby pravomocného skončení soudního řízení.
    \item Jestliže bylo rozhodnutí o vyloučení zrušeno, nebo pokud členská schůze anebo soud rozhodl, že námitky člena proti rozhodnutí o vyloučení jsou důvodné, platí, že členství v družstvu nezaniklo.
\end{enumerate}

%===============================================================
\paragraph{VIII. Orgány družstva}
\begin{enumerate}
    \item Orgány družstva jsou
    \begin{enumerate}[label=\alph*.]
        \item členská schůze,
        \item předseda.
    \end{enumerate}
    \item Členská schůze je nejvyšším orgánem družstva, tvořeným všemi členy družstva. Předsedu členská schůze volí. Předsedou může být jen člen družstva splňující podmínky podle zákona a těchto stanov. Zástupce právnické osoby zvolené do funkce člena představenstva musí splňovat požadavky a předpoklady pro výkon funkce stanovené zákonem pro samotného člena tohoto voleného orgánu.
    \item Funkční období předsedu je 5 let.
    \item Člen ucházející se o zvolení do funkce předsedy je povinen o okolnostech podle předchozího odstavce, týkajících se jeho osoby, předem členskou schůzi uvědomit a současně sdělit, zda nejsou dány skutečnosti, pro které podle zákona o obchodních korporacích nemůže být do funkce zvolen, nebo je podle právní úpravy dané zákonem o obchodních korporacích vyloučen z výkonu funkce člena statutárního orgánu nebo je dána na jeho straně jiná překážka funkce. Nastane-li v době trvání výkonu jeho funkce některá okolnost nebo skutečnost uvedená v předchozí větě, je povinen o ni neprodleně uvědomit členskou schůzi. Ustanovení tohoto odstavce platí také pro zástupce právnické osoby, která se uchází se o zvolení do funkce předsedy i pro zástupce právnické osoby, která je členem tohoto voleného orgánu.
    \item Předseda může z funkce odstoupit. Nesmí tak však učinit v době, která je pro družstvo nevhodná. Jestliže členská schůze nestanoví jiný okamžik, zaniká funkce předsedy do 1 měsíce.
    \item Předseda může být z funkce odvolán usnesením členské schůze. Funkce zaniká, neurčí-li členská schůze jiný okamžik, přijetím usnesení o odvolání z funkce.
    \item V případě odstoupení z funkce, odvolání nebo jiného ukončení funkce, zvolí členská schůze na místo předsedy jiného člena družstva. Končí-li funkční období, musí být členská schůze svolána za účelem volby předsedy na další funkční období tak, aby se konala před skončením dosavadního funkčního období.
\end{enumerate}

%===============================================================
\paragraph{IX. Účast člena na členské schůzi}
\begin{enumerate}
    \item Člen má právo účastnit se členské schůze osobně nebo v zastoupení. Plná moc pro zastupování na členské schůzi musí být písemná a musí z ní vyplývat, zda byla udělena pro zastoupení na jedné nebo na více členských schůzích. Nikdo nesmí být na jednání členské schůze zmocněncem více než jedné třetiny všech členů družstva, jinak platí, že nemá pro jednání na členské schůzi udělenu žádnou plnou moc.
    \item Členská schůze se může konat prostřednictvím oficiálního elektronického komunikačního kanálu, který umožňuje ověření totožnosti členů.
    \item Členská schůze rozhoduje usnesením přijímaným hlasováním členů družstva. Výkon hlasovacího práva člena lze omezit, vyloučit nebo pozastavit jen tehdy, stanoví-li tak zákon.
    \item Každý člen má při hlasování jeden hlas, jde-li o přijetí usnesení, kterým členská schůze rozhoduje o
    \begin{enumerate}[label=\alph*.]
        \item schválení poskytnutí finanční asistence,
        \item zrušení družstva s likvidací,
        \item přeměně družstva,
        \item vydání dluhopisů.
    \end{enumerate}
    \item Člen nemůže na členské schůzi vykonávat hlasovací právo
    \begin{enumerate}[label=\alph*.]
        \item rozhoduje-li členská schůze o námitkách tohoto člena proti rozhodnutí o jeho vyloučení z družstva,
        \item rozhoduje-li členská schůze o jeho odvolání z funkce, do níž byl zvolen,
        \item rozhoduje-li členská schůze o schválení poskytnutí finanční asistence ve vztahu k němu.
    \end{enumerate}
    \item Omezení výkonu hlasovacího práva se vztahuje i na každého člena, který ve smyslu zákona o obchodních korporacích jedná ve shodě s tím, kdo nemůže vykonávat hlasovací právo z důvodu uvedeného v předchozím odstavci tohoto článku.
\end{enumerate}

%===============================================================
\paragraph{X. Svolání členské schůze}
\begin{enumerate}
    \item Členská schůze se svolává pozvánkou odeslanou každému členu na adresu určenou pro doručování v oficiálním komunikačním kanálu uvedenou v seznamu členů nejméně 15 dnů přede dnem konání členské schůze. Pozvánka se uveřejňuje v této lhůtě též v oficiálním komunikačním kanálu, kde musí být přístupná každému členu družstva až do okamžiku konání členské schůze.
    \item Pozvánka na členskou schůzi musí obsahovat alespoň
    \begin{enumerate}[label=\alph*.]
        \item místo a dobu zahájení členské schůze; místo a doba zahájení členské schůze se určí tak, aby co nejméně omezovaly možnost člena se jí zúčastnit,
        \item označení, zda se svolává členská schůze nebo náhradní členská schůze,
        \item program členské schůze
        \item místo, kde se člen může seznámit s podklady k jednotlivým záležitostem programu členské schůze, pokud nejsou přiloženy k pozvánce.
    \end{enumerate}
    \item Má-li dojít ke změně stanov nebo k přijetí usnesení, jehož důsledkem je změna stanov, musí pozvánka obsahovat v příloze též návrh těchto změn nebo návrh usnesení.
    \item Předseda svolává členskou schůzi nejméně jednou za každé účetní období.
    \item Členská schůze svolaná předsedou k projednání řádné účetní závěrky se musí konat vždy nejpozději do 6 měsíců po skončení účetního období, za které je účetní závěrka sestavena.
    \item Předseda je povinen svolat členskou schůzi vždy, kdy je k tomu dán důležitý zájem družstva.
    \item Předseda je povinen bez zbytečného odkladu svolat členskou schůzi a navrhnout členské schůzi přijetí potřebných opatření, jestliže
    \begin{enumerate}[label=\alph*.]
        \item ztráta družstva dosáhla takové výše, že při jejím uhrazení ze zdrojů družstva by neuhrazená ztráta dosáhla výše základního kapitálu nebo to lze s ohledem na všechny okolnosti předpokládat, nebo
        \item družstvo se dostalo do úpadku nebo do hrozícího úpadku.
    \end{enumerate}
    \item Předseda je povinen bez zbytečného odkladu svolat členskou schůzi také tehdy, jestliže jej o to požádalo alespoň 20 \% členů družstva.
    \item Jestliže není členská schůze svolána do uplynulé lhůty pro svolání předsedou, může členskou schůzi svolat a všechny úkony s tím spojené činit osoba k tomu zmocněná všemi členy, kteří o svolání členské schůze požádali.
    \item Na žádost alespoň 20 \% členů družstva je předseda povinen zařadit těmito členy určenou záležitost do programu uvedeného na pozvánce na členskou schůzi.
\end{enumerate}

%===============================================================
\paragraph{XI. Rozhodování členské schůze}
\begin{enumerate}
    \item Členskou schůzi řídí osoba, která za podmínek podle těchto stanov členskou schůzi svolala. Na návrh toho, kdo členskou schůzi svolal, může být jejím řízením pověřen i jiný člen družstva.
    \item Členská schůze jedná podle programu uvedeného v pozvánce. Záležitosti, které nebyly zařazeny do programu uvedeného v pozvánce na členskou schůzi, lze na členské schůzi projednat jen za účasti a se souhlasem všech členů družstva.
    \item Členská schůze schopna se usnášet, pokud je přítomna většina všech členů.
    \item Členská schůze se usnáší většinou hlasů přítomných členů, nevyžaduje-li zákon o obchodních korporacích nebo tyto stanovy vyšší počet hlasů.
    \item Elektronické hlasování je považováno za písemné hlasování mimo shromáždění, kde je potřeba nadpoloviční většina hlasů všech členů, nikoliv jen přítomných.
    \item Při posuzování schopnosti členské schůze se usnášet a při přijímání usnesení se nepřihlíží k přítomnosti a hlasům členů, kteří nemohou podle zákona o obchodních korporacích v případech uvedených v článku IX. odstavci 4 těchto stanov vykonávat hlasovací právo.
    \item Hlasuje se veřejně, pokud se v jednotlivých případech členská schůze předem neusnese na tajném hlasování.
    \item Tajné hlasování o návrhu na změnu stanov nebo návrhu jiného rozhodnutí, jehož důsledkem je změna stanov, se zakazuje.
    \item Jestliže má být o některé záležitosti rozhodováno tajným hlasováním, zvolí členská schůze na návrh osoby oprávněné řídit členskou schůzi veřejným hlasováním 1 z přítomných členů, který rozdá členům hlasovací lístky a z hlasovacích lístků odevzdaných do schránky určené osobou oprávněnou členskou schůzi řídit zjistí a oznámí členské schůzi výsledek hlasování.
    \item Není-li členská schůze schopna se usnášet, svolá bez zbytečného odkladu ten, kdo svolal původně svolanou členskou schůzi, náhradní členskou schůzi se stejným programem, a to stejným způsobem jako původně svolanou členskou schůzi a samostatnou pozvánkou, jestliže je stále potřebné, aby se náhradní členská schůze konala. Není-li však schopna se usnášet členská schůze, která byla svolána na žádost členů oprávněných podle těchto stanov požadovat její svolání a žádost nebyla vzata zpět, svolá způsobem uvedeným v předchozí větě náhradní členskou schůzi ten, kdo členskou schůzí svolal, vždy.
    \item Záležitosti, které nebyly zařazeny do navrhovaného programu řádné členské schůze, lze na náhradní členské schůzi rozhodnout jen tehdy, jsou-li přítomni a projeví-li s tím souhlas všichni členové družstva.
    \item Náhradní členská schůze je schopna se usnášet bez ohledu na počet přítomných členů.
    \item Podrobnosti o způsobu svého jednání upraví podle potřeby členská schůze usnesením.
\end{enumerate}

%===============================================================
\paragraph{XII. Působnost členské schůze}
\begin{enumerate}
    \item Členská schůze
    \begin{enumerate}[label=\alph*.]
        \item rozhoduje o využití fondu na následující čtvrtletí,
        \item schvaluje řádnou, mimořádnou nebo konsolidovanou účetní závěrku, popřípadě mezitímní účetní závěrku,
        \item rozhoduje o rozdělení zisku nebo úhradě ztráty na konci každého kvartálu,
        \item mění stanovy, nedochází-li k jejich změně na základě jiné právní skutečnosti,
        \item volí a odvolává předsedu,
        \item určuje výši odměny předsedy,
        \item schvaluje smlouvu o výkonu funkce,
        \item schvaluje poskytnutí finanční asistence,
        \item rozhoduje o námitkách člena proti rozhodnutí o jeho vyloučení,
        \item rozhoduje o vydání dluhopisů,
        \item rozhoduje o přeměně družstva,
        \item schvaluje smlouvu o tichém společenství a její změnu a zrušení,
        \item rozhoduje o zrušení družstva s likvidací,
        \item volí a odvolává likvidátora a rozhoduje o jeho odměně,
        \item schvaluje zprávu likvidátora o naložení s likvidačním zůstatkem,
        \item vyslovuje souhlas se smlouvou o vypořádání újmy vzniklé družstvu porušením péče řádného hospodáře, uzavíranou s povinnou osobou,
        \item vykonává působnost kontrolní komise v rozsahu vymezeném zákonem o obchodních korporacích,
        \item rozhoduje o dalších otázkách, které zákon nebo stanovy svěřují do její působnosti.
    \end{enumerate}
    \item Členská schůze si může vyhradit do své působnosti rozhodování o dalších otázkách, které zákon ani stanovy do její působnosti nesvěřují, pokud nejde o záležitosti patřící podle zákona o obchodních korporacích do působnosti předsedy. O záležitosti, kterou si členská schůze vyhradí do své působnosti, nelze na téže členské schůzi rozhodovat, pokud nejsou na této členské schůzi přítomni všichni členové družstva a současně všichni z nich nevysloví souhlas s tím, že se tato záležitost bude projednávat na této členské schůzi.
\end{enumerate}

%===============================================================
\paragraph{XIII. Zápis o průběhu členské schůze}
\begin{enumerate}
    \item O průběhu členské schůze pořídí zápis ten, kdo ji svolal. Zápis o průběhu členské schůze musí obsahovat
    \begin{enumerate}[label=\alph*.]
        \item údaje označující orgán nebo osobu, která členskou schůzi svolala,
        \item místo konání, dobu zahájení a dobu ukončení členské schůze, s uvedením, zda byla svolaná jako členská schůze nebo náhradní členská schůze, a jde-li o náhradní členskou schůzi, i dobu a místo, kde se měla konat členská schůze původně svolaná,
        \item program jednání,
        \item přijatá usnesení, přičemž u každého z usnesení musí být údaje o počtu členů přítomných při hlasování osobně nebo v zastoupení, o počtu jejich hlasů a počtu hlasů, které odevzdali pro přijetí usnesení; jmenovitě se zde uvede také každý člen, který nemohl při hlasování vykonávat hlasovací právo a k jehož přítomnosti i hlasům se proto nepřihlíží,
        \item znění každé z případných námitek člena s údajem o jménu člena, který námitku uplatnil,
        \item jméno každého člena družstva, který nehlasoval pro změnu stanov, jestliže členská schůze přijala usnesení o změně stanov nebo usnesení, jehož důsledkem je změna stanov.
    \end{enumerate}
    \item Zápis podepíše ten, kdo členskou schůzi svolal.
    \item Člen družstva má právo na vydání digitální kopie zápisu o průběhu členské schůze. 
    \item Notářský zápis, kterým se osvědčuje usnesení členské schůze, se uloží jako příloha zápisu o průběhu členské schůze.
\end{enumerate}

%===============================================================
\paragraph{XIV. Předseda}
\begin{enumerate}
    \item Předseda rozhoduje jako statutární orgán družstva o všech záležitostech družstva, které nesvěřuje zákon nebo tyto stanovy jinému jeho orgánu.
    \item Předsedovi přísluší obchodní vedení družstva.
    \item Předseda plní usnesení členské schůze, není-li v rozporu s právními předpisy.
    \item Předseda odpovídá za operativní nakládání s fondem v souladu s usnesením členské schůze.
    \item Předseda zajišťuje řádné vedení seznamu členů, účetnictví, předkládá členské schůzi ke schválení účetní závěrku a v souladu se stanovami také návrh na rozdělení zisku nebo úhradu ztráty.
    \item Předseda zastupuje družstvo jako člen jeho statutárního orgánu.
\end{enumerate}

%===============================================================
\paragraph{XV. Závěrečná a přechodná ustanovení}
\begin{enumerate}
    \item Družstvo je povinno vydat tyto stanovy každému členovi.
    \item Dojde-li ke změně stanov na základě právní skutečnosti, předseda vyhotoví úplné znění stanov bez zbytečného odkladu poté, co se o této skutečnosti dozví. Úplné znění stanov uveřejní předseda na informační desce. Jestliže o to člen požádá, je předseda povinen mu úplné znění stanov vydat.
    \item Schválením tohoto znění stanov se družstvo podřizuje zákonu o obchodních korporacích jako celku. Údaj o tom se zapíše do obchodního rejstříku.
    \item Toto znění stanov nabývá účinnosti dnem zveřejnění zápisu o podřízení se družstva zákonu o obchodních korporacích jako celku v obchodním rejstříku.
\end{enumerate}

%===============================================================
\chapter{Web}
%===============================================================
The website is accessible at \href{https://grafit.games/}{\textbf{https://grafit.games/}}

A copy of the source code is included in the thesis repository.
The full repository hosting the github page can be found at:

\href{https://github.com/CernaSam/grafit-games-web}{\textbf{https://github.com/CernaSam/grafit-games-web}}

The website takes on the grafit design language. The original repository can be found at:

\href{https://github.com/CernaSam/grafit-games-web}{\textbf{https://github.com/grafitctu/grafitctu.github.io}}
