%===============================================================
\chapter{Foreign Solutions}
%===============================================================

\begin{chapterabstract}
    Objective:
    This chapter examines the core functions, structures, and impact of university-based incubators—both general and game-focused—in supporting student and graduate entrepreneurship, with a particular focus on how these incubators facilitate the transition from idea to viable business in the context of game development.
    Context:
    University incubators have become essential engines of innovation, offering mentorship, funding, workspace, and industry connections to early-stage startups. They serve as bridges between academia and industry, helping students and researchers overcome barriers to commercialization. Specialized game incubators have recently emerged to address the unique needs of game development teams, offering targeted support in design, production, and market entry. Pre-incubation programs and sector-specific initiatives further expand the landscape of support, allowing students to test ideas before formal company formation.
    Action:
    The chapter surveys prominent incubator models and their offerings, from mentorship and funding to networking and access to facilities. It highlights international and Czech examples—including Gamebaze in Brno, Sweden Game Arena, and NYU Game Center Incubator—while also exploring pre-incubation and specialized support for game startups. The analysis covers how these incubators structure their programs, select participants, and foster multidisciplinary collaboration, as well as the challenges and best practices in supporting student ventures from ideation through launch.
    Results:
    Readers will gain a comprehensive understanding of the functions and benefits of academic incubators, the distinctive features of game-focused programs, and the value of pre-incubation. This knowledge equips students, educators, and policymakers to better leverage incubator resources, adapt support structures to the needs of game startups, and strengthen the entrepreneurial ecosystem within universities and beyond.
\end{chapterabstract}

University-based incubators play a pivotal role in transforming innovative ideas into successful businesses, particularly for students and recent graduates navigating the early stages of entrepreneurship. These incubators offer a suite of essential services—mentorship from industry veterans, structured training programs, access to funding, networking opportunities, and subsidized facilities—that collectively lower the barriers to entry for aspiring founders. By connecting participants with investors, corporate partners, and alumni networks, incubators accelerate business development and enhance the prospects for long-term success.
As the landscape of student entrepreneurship evolves, so too have the models of incubation. Many universities now offer pre-incubation programs that allow students to test and refine their ideas in a supportive environment before committing to formal company formation. These programs emphasize multidisciplinary teamwork, entrepreneurial mindset development, and practical business skills, making them especially valuable for students from diverse academic backgrounds.
The emergence of specialized game incubators marks a significant development in this space. Game development ventures face unique challenges—such as rapid product cycles, complex production pipelines, and fierce market competition—that require tailored support. Initiatives like Gamebaze in Brno, Sweden Game Arena, and the NYU Game Center Incubator provide targeted mentorship, industry connections, and training specifically designed for game startups. These programs foster collaboration between students, faculty, and industry professionals, helping teams navigate the journey from prototype to market-ready product.
Despite their many advantages, incubators also encounter challenges, including funding sustainability, scaling support as demand grows, and ensuring access to mentors with relevant expertise. Nevertheless, their impact is evident in the growing number of student-led ventures, successful alumni, and strengthened academic-industry ties.
This chapter explores the multifaceted functions of traditional and game-focused incubators, the role of pre-incubation in nurturing early-stage ideas, and the evolving best practices that underpin effective support for student entrepreneurship. By understanding these mechanisms, universities and stakeholders can better position themselves to foster innovation, support commercialization, and drive economic and social impact through student-led ventures.

%===============================================================
\section{Prominent Traditional Incubator Functions}
Major university-based incubators offer a variety of functions and services to support startups.

Incubators provide mentorship from industry experts, successful entrepreneurs, and alumni and run workshops or other structured programmes to develop entrepreneurial skills. The Macquarie University (in Sydney)
Incubator’s mentor programme directly matches founders with experts to share insights, help them navigate the journey and accelerate their success. 

Many incubators provide direct funding or help startups secure grants and investments. The UnternehmerTUM offers €5 000 for prototyping in their incubator, €25 000 of project budget in their accelerator and up to €250 000 in total funding. Cambridge Enterprise invested £6.47 million in 37 spinout companies (2023-24). SETsquared helped raise £5bn in investments.

Incubators facilitate connections with investors, corporate partners, and other startups. Over 6 500 companies have participated in the SETsquared programmes and Startup Autobahn partners with companies like Porsche and Daimler.

They also tend to provide free or subsidized facilities. Lund University’s VentureLab offers free office space, coffee and fruit. RWTH Innovation provides access to research facilities. 

Some incubators specialize in specific industries or technologies, support sustainability and social impact. Polihub focuses on deep tech startups, Wyss Zurich emphasizes regenerative medicine and robotics and Tartu University CDL-Estonia specializes in digital government and cybersecurity. EIT Climate-KIC supports climate-related startups.

Incubators also promote global competitions and highlight successful alumni as role models for incoming students and new startups. Cambridge Enterprise supported Raspberry Pi and the alumni of Yes!Delft include Ampelmann, a maritime tech company. EUT+ Incubation Program’s participants get to compete in the EUt+ Finals against the best teams from EUt+ campuses.

% %===============================================================
\section{Pre-Incubation}
Many students seek the opportunity to test and develop their ideas in a supportive environment before establishing a formal business entity.

The EIT Digital Venture programme takes entrepreneurs "from idea to investment in less than a year" and is available across 24 European countries. The programme provides financial support (up to €30 000), MVP and business development assistance from experts and a direct connection to Europe's innovation ecosystem all without requiring immediate legal registration of a company.

The Italian Ministry of Education has financed university laboratories (named CLabs) with the aim of developing an entrepreneurial mindset and competences. CLabs offer a new view on university-based business idea incubators at a national level. The OECD has recently considered them as one of the best ways of supporting student entrepreneurship and innovation. They focus primarily on developing entrepreneurial competencies and ideas. These programmes "[...] act as pre-incubators or pre-accelerators that are designed to help a growing number of university students from different backgrounds interact and develop their entrepreneurial ideas in a safe and creative environment." They emphasize motivation and multidisciplinary teamwork over formal business directing, with a selection process that values student enthusiasm more than their grades or the initial quality of their business ideas.

The Technology Incubation program in the Czech Republic admits young startups, spin-off companies, students and scientific projects with commercial potential - allowing participation without immediate company formation. The program focuses on supporting startups rather than established companies, with eligibility based on the innovation potential of the idea rather than legal status.

Academic Business Incubators (by Business Centre Club) allow young entrepreneurs to save time and effort associated with establishing a company. Students who join the incubators can start their own independent venture that is formally a unit of the organisation. Owners of such companies have no obligation to pay social security contributions however are also not entitled to unemployment aid programmes.

% %===============================================================
\section{Prominent Game Incubators}
Universities worldwide have established game incubators to support students and graduates in developing their ideas into successful ventures. This is an overview of some of the prominent university-affiliated game incubators in Czechia and beyond.

Gamebaze is a joint initiative in Brno (between Game Cluster, JIC/KUMST and the GameDev Area) supporting gaming-related startup projects. This incubator is part of Czechia's robust gaming education ecosystem that also includes partnerships with local studios like Warhorse Studio.

Sweden Game Arena, located in Skövde, is a world-leading hub for game development. It unites a game development bachelor's and master's programme with a successful incubator under one umbrella. This ecosystem supports students by offering practical collaboration opportunities with a large number of game companies and access to industry events such as the Sweden Game Conference.

Game Hub Denmark operates in three cities - Grenaa, Aalborg, and Viborg. It includes facilities like the BizHub for secondary school students, Aalborg University Game Hub for entrepreneurs, and Roof Creative Industries Incubator in Viborg - part of one of the best animation schools in the world. The initiative also collaborates internationally - in typically EU-funded development projects - to expand opportunities for game startups.

GameBCN in Barcelona, Spain is a 5 months long programme not part of a university. It catered to teams from all over the world on the condition that they relocate to Barcelona. The programme includes 90 hours of general training focused on production, marketing and business. The participating teams can get feedback about their projects and strategy from industry professionals on monthly meetings. No equity is taken in the companies that are selected for the programme.

Carbon Incubator from Bucharest, Romania caters primarily to indie developers from Eastern Europe. The programme (not part of a university) asks for revenue share of games launched by participating companies. A company that received incubation services is required to give up a revenue share of 10\%. The share rises to 20\% for companies in their acceleration programme and to 30\% for companies receiving funding.

The NYU Game Center Incubator’s programme begins with in-person workshops and coworking sessions in Brooklyn, transitioning to remote collaboration for the remainder of the year. Participants receive \$15,000 in funding per team, mentorship from an executive producer, access to industry workshops, and one-on-one guidance from an advisory board of game industry professionals. The program partners with major industry players like Sony and Microsoft while maintaining a commitment to inclusivity and diversity in game development.
