%===============================================================
\chapter{Our Games and Commercialization Support Options}
%===============================================================

\begin{chapterabstract}	
    Objective:
    This chapter provides an overview of game development education, opportunities, and support at the Faculty of Information Technology, Czech Technical University in Prague (FIT CTU), highlighting how the faculty fosters student creativity and prepares aspiring developers for both technical excellence and entrepreneurial success.
    Context:
    FIT CTU offers a robust Informatics program with a specialization in computer graphics, combining theoretical foundations with hands-on coursework in areas such as programming, visualization, and user interface design. The faculty regularly organizes events like Game Jams and supports a range of student-driven game projects, resulting in a diverse portfolio of innovative games. While entrepreneurial support is not directly embedded within FIT CTU, students benefit from university-wide initiatives, incubators, and career development programs that help translate creative projects into viable ventures.
    Action:
    The chapter details the educational pathways available at FIT CTU for game development, showcases notable student projects, and outlines the ecosystem of events and support services that nurture both technical and entrepreneurial skills. It also examines the evolving landscape, including the upcoming Applied Informatics program with a dedicated Game Development specialization, and the role of external resources such as the InQbay incubator and the CTU Career Centre.
    Results:
    Readers will gain insight into how FIT CTU equips students with the knowledge, practical experience, and support necessary to create original games and potentially launch them as commercial products. The chapter underscores the importance of integrating technical education with entrepreneurial guidance to maximize the impact of student innovation in the game industry.
\end{chapterabstract}

Game development at FIT CTU is a dynamic and integral part of the faculty’s academic and extracurricular offerings. The Informatics program, with its computer graphics specialization, provides students with a rigorous foundation in both the theory and practice of game creation. Courses range from computer graphics programming and modern visualization technologies to multimedia applications and user interface design, ensuring that students acquire a comprehensive skill set relevant to today’s game industry.
The faculty actively encourages hands-on learning through events such as the FIT CTU Game Jam, where students collaborate under time constraints to produce innovative games, often with guidance from industry experts. These events, alongside coursework, research groups, and independent projects, create a vibrant environment where aspiring developers can experiment, learn, and showcase their talents.
Over the years, FIT CTU students have produced a variety of original and technically impressive games, many of which have gained recognition for their creativity, storytelling, and innovative mechanics. Examples like Encore!, Liminal!, Escape from Brno, and Subject 42 demonstrate the breadth of student achievement and the faculty’s commitment to nurturing both individual and team-based projects.
While FIT CTU itself does not provide direct entrepreneurial support, students have access to university-level resources such as the InQbay incubator and the Career Centre, which offer coaching, workshops, and connections to industry professionals. These initiatives help bridge the gap between academic achievement and real-world application, supporting students who wish to commercialize their games or pursue entrepreneurial careers.
The Importance of Understanding the Full Game Development Pipeline
To fully leverage the opportunities at FIT CTU, students must grasp the entire game development process—from initial concept and design through production, testing, launch, and beyond. Each stage presents unique challenges and requirements, from technical polish and teamwork to legal, financial, and marketing considerations12. For student projects to transition successfully from classroom or competition to commercial release, a holistic understanding of this pipeline is essential. By integrating technical education with entrepreneurial support, FIT CTU empowers students not only to create compelling games but also to navigate the complexities of bringing their creations to a broader market.

%===============================================================
\section{Game Development at FIT CTU}
The Faculty of Information Technology at the Czech Technical University in Prague (FIT CTU) offers multiple courses and events dedicated to game development. 

There is currently one study programme at the Faculty of Information Technology - Informatics. It offers multiple specializations, one of which is the computer graphics specialization, which provides a combination of theoretical foundations (with courses such as Computer graphics programming, Modern Visualisation Technologies, Machine vision and image processing) and hands-on experience (with courses such as Multimedia and Graphics Applications, Programming of Graphic Applications, User Interface Design).

A new study programme is being prepared at FIT CTU that will be called Applied Informatics and have three brand new specializations - Game Development, Graphics and Computer Vision. It will provide a higher focus on hands-on experience and more study places in the game development domain. A broad spectrum of games is expected to be created that could benefit from entrepreneurial support.

Some of the other CTU courses provide the option to develop games as well (notably Team Software Project).

The FIT CTU Game Jam challenges students to create a computer game during a 48 hour period over a prolonged weekend. The students are given an assignment at the beginning of the event, compete either individually or in a team and are advised by industry experts.

Several study paths at FIT CTU lead to the creation of games. Whether through specialized courses, events, research groups, or student-driven projects, the faculty provides an environment for aspiring game developers to develop their skills and bring their visions to life.

%===============================================================
\section{Games Created at FIT CTU}
Over the years, students at the FIT CTU have created a wide range of original, comical and technically impressive games. These projects often combine creative storytelling and original game mechanics and are developed as part of coursework, bachelor’s or master’s theses, Game Jams or independent student initiatives.

Encore! developed during the 2024 Game Jam by Belonzik and TheMultiplexx is a dueling  card game with an outstanding attention to detail following the theme of death. It has excellent graphics, audio, story and even narration.

Liminal! by HyperCubic Studio was developed during the 2024 Game Jam too. It is a short platform puzzle game, where the player is captured in a maniac’s TV show - becoming laughing stock for the viewers. The game’s colourful visuals are beautiful, cohesive and extensively polished. It has original mechanics and is full of details in the sounds, menu items and dialogue.

Escape from Brno was created during the 2022 Game Jam by Trampod, SharpFoxDev, benjaminhejl, leia12321 and VAHAnima. The game has won the popular vote among the competing developers. It is a classic side-scroller and dodger, has cohesive graphics, humorous sound design, and the gameplay itself is easy to grasp and feels natural.

Subject 42 was created by LukyDrum during the 2023 GameHack. The player takes on the role of a robot’s AI, solving puzzles and guiding it through three intriguing levels. Accompanying are a fitting sound-track and a voice narrating the story and sarcastically commenting on the player’s performance.

%===============================================================
\section{Local Student Entrepreneurship Support}
FIT CTU does not directly provide entrepreneurial support. It manages cooperation with industry partners and gathers grants for research labs. The university takes over the responsibility of arranging access to dedicated incubators and coaching centers. These initiatives help students refine their business ideas and develop their skills.

A key programme supporting student, phd and faculty entrepreneurs at CTU is InQbay. It offers individual coaching, workshops, tutorials, networking events and connects members to legal, tax and marketing consultants.

The career centre of CTU offers an 8 weeks long course on entrepreneurship. It goes over ideation, finding a target group, working with a lean canvas, creating a business plan, pitching a project, gathering feedback and gaining a mentor or an investor.