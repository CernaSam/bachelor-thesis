%===============================================================
\chapter{University-Based Start-Up Facilitation Programmes}
%===============================================================

\begin{chapterabstract}
    University incubators have become essential engines of innovation, offering mentorship, funding, workspace, and industry connections to early-stage start-ups. They bridge academia and industry, helping students and researchers overcome barriers to commercialization. Specialized game incubators have recently emerged to address the unique needs of game development teams. Pre-incubation programmes further expand the available support options, allowing students to test ideas before formal company formation. This chapter surveys prominent incubator models and their offerings in the Czech Republic and internationally. The analysis covers how the programmes are structured, select participants, and propone collaboration, as well as the challenges and best practices in supporting student ventures from ideation through launch.    
\end{chapterabstract}

Game development ventures face unique challenges - such as rapid product cycles and fierce market competition - making commitment to a commercial project difficult. Tailored initiatives - providing targeted mentorship, initial capital, and industry connections - could help teams accelerate the journey from prototype to market-ready product and lower the barriers to entry for aspiring founders.

%===============================================================
% \section{Start-Up Incubators and Accelerators}
\textbf{Start-up incubators} help develop and refine high-potential business ideas. They usually operate locally and provide resources over a span of one to five years. Incubators tend to offer product development guidance, on-demand co-working space, legal consultation, networking opportunities and mentorship.

\textbf{Start-up accelerators} are short, intensive programmes for early- or mid-stage founders. Accelerators are more structured than incubators and outline specific steps to create a scalable business. They often have an alumni and investor network and offer funding in return for stake in the company. Participants usually go through intensive mentorship from industry leaders on fundraising, product development, and growth marketing.

%===============================================================
\section{Prominent Incubator Functions}
Major university-based incubators offer a variety of functions and services. Primarily, they provide some kind of mentorship - from industry experts, successful entrepreneurs, and or alumni. They run workshops or tutorials to help develop entrepreneurial skills and other relevant competencies. For example The Macquarie University (in Sydney) Incubator’s mentor programme directly matches founders with experts to help them navigate the journey and share their insights.

Incubators also tend to provide free or subsidized facilities. RWTH Innovation grants access to research facilities. Lund University’s VentureLab even offers free office space, coffee and fruit.

Many incubation programmes provide direct capital or help start-ups secure grants and investments. The UnternehmerTUM offers €5~000 for prototyping in their incubator, a €25~000 project budget in their accelerator and up to €250~000 in total funding. Cambridge Enterprise invested £6.47~million in 37 spinout companies (from 2023 to 2024). SETsquared helped raise £5~billion in investments.

Incubators typically facilitate connections with investors, corporate partners, and other starting ventures. For example, over 6~500 companies have participated in the SETsquared programmes and Startup Autobahn partners with companies like Porsche and Daimler.

Some incubators specialize in specific industries or technologies, support sustainability and social impact. Polihub focuses on deep tech start-ups, Wyss Zurich emphasizes regenerative medicine and robotics and Tartu University CDL-Estonia specializes in digital government and cybersecurity. EIT Climate-KIC supports climate-related start-ups.

Moreover, incubators promote global competitions and highlight successful alumni as role models for incoming students and upcoming entrepreneurs. Cambridge Enterprise supported Raspberry Pi and the alumni of Yes!Delft include Ampelmann, a maritime tech company. EUT+ Incubation Program’s participants get to compete in the EUt+ Finals against the best teams from EUt+ campuses.

%===============================================================
\section{Incubation of Early-Stage Venture}
Many students seek the opportunity to test and develop their ideas in a supportive environment before establishing a formal business entity.

The EIT Digital Venture programme takes entrepreneurs from an idea to investment-seeking endeavours in less than a year. It is available across 24 European countries and offers financial support (up to €30~000), MVP and business development assistance from experts and a direct connection to Europe's innovation ecosystem all without requiring immediate legal registration of a company.

CLabs were founded by the Italian ministry of education with the aim of developing an entrepreneurial mindset on a national level. The OECD recognized them as one of the best ways of supporting student entrepreneurship and innovation. They "[...] act as pre-incubators or pre-accelerators that are designed to help a growing number of university students from different backgrounds interact and develop their entrepreneurial ideas in a safe and creative environment." Emphasizing motivation and multidisciplinary teamwork over formal business directing, students are selected in a process valuing enthusiasm more than grades or the initial quality of a business idea.

The Technology Incubation program in the Czech Republic admits young start-ups, spin-off companies, students, and scientific projects with commercial potential - allowing participation without the immediate need for company formation. The programme selects participants based on the innovation potential of their idea rather than operational state.

Academic Business Incubators (by Business Centre Club) allow young entrepreneurs to save time and effort associated with establishing a company. Students who join the incubators can start their own independent venture that is formally a unit of the organisation. Owners of such companies have no obligation to pay social security contributions however are also not entitled to unemployment aid programmes.

%===============================================================
\section{Prominent Game Incubators}
Game-focused incubators are beginning to emerge worldwide, aiming to nurture students and graduate creations into successful ventures. 

Gamebaze is an incubator formed by a joint initiative (between Game Cluster, JIC/KUMST and the GameDev Area) in Brno. It supports gaming-related start-up projects and is part of Czechia's robust gaming education ecosystem. It also facilitates partnerships with local studios like Warhorse Studio.

Sweden Game Arena - located in Skövde - unites a game development bachelor's and master's programme with a successful incubator under one umbrella. This ecosystem supports students by offering practical collaboration opportunities with a large number of game companies and access to industry events such as the Sweden Game Conference.

Game Hub Denmark operates in three cities - Grenaa, Aalborg, and Viborg. It includes facilities like the BizHub for secondary school students, Aalborg University Game Hub for entrepreneurs, and Roof Creative Industries Incubator in Viborg - part of one of the best animation schools in the world. The initiative also collaborates internationally - in typically EU-funded development projects - to expand opportunities for game start-ups.

When it comes to financing, The NYU Game Center Incubator grants its participants \$15~000 per team. The programme begins with in-person workshops and coworking sessions, transitioning to remote collaboration for a remainder of the year. The mentors include producers and industry professionals and the incubator partners with major industry players like Sony and Microsoft.

GameBCN in Barcelona, Spain is a 5 months long programme accommodating teams from all over the world. It includes 90 hours of training - on production, marketing, and business - and feedback from industry professionals on monthly meetings. The selected companies are not required to give up equity.

On the other hand, Carbon Incubator from Bucharest, Romania - catering primarily to indie developers from Eastern Europe - requires their incubated companies to give up 10\% of revenue. The share rises to 20\% in their acceleration programme and to 30\% for companies receiving funding.
