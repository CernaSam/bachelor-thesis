%===============================================================
\chapter{Game Development Processes}
%===============================================================

\begin{chapterabstract}
	The game development process is a series of interconnected stages taking creatives from an idea to a commercial product. Understanding of the process allows a developer to make informed decisions, a supportive body to provide targeted assistance and an investor to assess the state and outlook of a project. The chapter breaks down each component of the process - from planning, pre-production and production to testing, launch and post-launch activities. It especially focuses on the launch phase - requiring a distinct set of competencies from marketing and distribution to niche legal and financial expertise, often underrepresented in the technically and creatively oriented developer teams.
\end{chapterabstract}

Game development is not a single, linear task but a series of interconnected stages, each with its own distinct goals and requirements. A thorough understanding of the entire process is essential for any student team aiming to successfully bring their game to the release stage - allowing them to anticipate challenges, allocate resources effectively, and make informed decisions at every step, from concept to completion.

%===============================================================
\section{Game Development Stages}
Game development is the process of designing, creating, and releasing video games. It includes writing, sound design, project management, programming and more. The process can be divided into distinct stages that focus on different aspects of the final product.
\cite{bramble_7-stages, rocket_6-stages}

%---------------------------------------------------------------
\paragraph{Planning Stage}
In the initial stage, game developers choose the genre that fits their vision the best, select viable art styles and gameplay mechanics, plan the game’s structure, content, and more. Changing, cutting or replacing later on can be straightforward in some aspects of the game and very challenging in others. Such aspects must be decided early on.
\cite{bramble_7-stages, rocket_6-stages}

%---------------------------------------------------------------
\paragraph{Pre-Production Stage}
The pre-production stage of game development requires artists, writers and designers to finalise important decisions. Feasibility, practicality and the worth of different design aspects is considered. Will the game be fun to play and appealing to look at? Will it work properly, or do some technical limitations need to be taken into account?
\cite{bramble_7-stages, rocket_6-stages}

%---------------------------------------------------------------
\paragraph{Production Stage}
After the decision-making, production of the game can start. It is at this stage when most of the code is written, levels are designed, game mechanics are tested, models, textures and visual elements start to appear.
\cite{bramble_7-stages, rocket_6-stages}

%---------------------------------------------------------------
\paragraph{Testing Stages}
Some form of internal testing is done throughout the entire process. Before the game is finalised however, developers tend to release test versions. This practice can be roughly divided into alpha and beta.

The \textbf{alpha} version of the game already has the key mechanics and allows developers to assess playability. It might have placeholders for characters, surroundings or lack music. It is used for internal\footnote{refered to as closed} testing between staff members but can in some cases be available\footnote{refered to as open} to selected, passionate fans willing to help developers with playtesting.

The \textbf{beta} version follows alpha. The game still requires a lot of work at this point but elements such as the environment and characters are approaching their final form. There still might be bugs present, glitches and exploits that need fixing, performance optimization required, and details missing. The game mechanics may still need to be balanced and server stability tested.\footnote{Betas can be open or closed too.}
\cite{bramble_7-stages, rocket_6-stages, esler_viable-games}

%---------------------------------------------------------------
\paragraph{Launch Stage}
During the launch stage the game is made available for the public to play. This stage requires understanding the target market, audience, selecting a distribution channel, creating a strategy and advertising. Additional support for players might be provided and feedback gathered.
\cite{bramble_7-stages, rocket_6-stages, esler_viable-games}

%---------------------------------------------------------------
\paragraph{Post-Launch Stage}
After the initial publication, developers might want to release updates, patch bugs or even add new content, either as a free update or in the form of a purchasable extension. Continuation of a successful product allows it to extend its lifespan and provides a long-term fanbase.
\cite{bramble_7-stages, rocket_6-stages}

%===============================================================
\section{Launch Process In Detail}
While the students at the Faculty of Information Technology at the Czech Technical University (FIT CTU) often excel at designing and programming games, the launch phase is where many projects struggle. Unlike development, which follows a structured technical process, launching a game involves a complex and often unfamiliar set of tasks, from marketing and distribution to niche legal and financial specialties, budget planning, fundraising, assessing copyright protection, trademarks, creating contracts and more. A successful launch requires careful planning, strategic timing, and an understanding of distribution platforms and promoting. Many of these steps are not immediately obvious but can determine whether a game finds an audience or gets lost in an oversaturated market.

%---------------------------------------------------------------
\subsection{Structural Requirements}
Ensuring a smooth and successful launch requires meeting critical structural requirements that impact a game’s performance, security, and compliance. Overlooking these factors can lead to negative user experiences, security vulnerabilities, and even regulatory consequences.

Before being released to the public, games need to be assessed and extensively tested to ensure stability and playability. The first thing a player interacts with is the UI - a launch screen or app - which therefore needs to be optimized. Settings such as the resolution, window size, language, subtitles and key bindings need to work properly. Accessibility and support options need to be tested and credits/end game screens polished.\cite{silva_guide-to-release}

For games with online components, a reliable server infrastructure is crucial. Poor server performance can lead to lag or disconnects during traffic surges. Optimizing configuration, considering scalability and running stress tests before launch helps identify potential bottlenecks.\cite{sentika_how-to-optimize}

Major gaming platforms, from Steam to PlayStation, Xbox and mobile app stores, have specific technical requirements. Failing to meet performance specifications or file size limitations can lead to rejection, or post-launch repercussions.\cite{apple-ios-apps}

Collecting and storing personal player data is best avoided in the case of small student-led projects. If collecting data is necessary, developers must comply with regional regulations such as the General Data Protection Regulation (GDPR)\cite{eu_gdpr} in Europe and the California Consumer Privacy Act (CCPA)\cite{doj_ccpa} in the USA. These regulations require data to be anonymized, encrypted, safely stored and access to it minimized. Non-compliance can lead to severe penalties.\cite{silva_guide-to-release}

%---------------------------------------------------------------
\subsection{Operational Requirements}
A successful game launch relies on careful coordination of tasks and resources. While technical readiness is crucial, the operational aspects of the launch determine how smoothly the transition from development to release unfolds.

Best practices include creating a detailed timeline for launch-related activities and setting a realistic launch date. Mapping critical milestones helps teams avoid last-minute chaos.\cite{edgegap_pre-launch-list}

Effective launch coordination requires collaboration between developers, marketers, community managers, and support staff. Clearly defining individual responsibilities and establishing a contingency plan can help prevent miscommunication or overlooking tasks. A pre-launch meeting can align all team-members and prepare them for launch day chaos.\cite{palmer_planning-launch, edgegap_pre-launch-list}

Preparing announcements for player communication channels (e.g., Discord, Reddit) to address potential issues or provide updates can promptly address player inquiries and provide updates.\cite{palmer_planning-launch, edgegap_pre-launch-list}

%---------------------------------------------------------------
\subsection{Legal Requirements}
From a legal standpoint, publishing a game entails compliance with intellectual property laws, consumer rights regulations, and distribution agreements. Intellectual Property (IP) protection, including copy rights, trademarks, and patents, safeguards creators’ work. IP applies to a finished game, but might also restrict use of assets such as code, art, music and branding.\cite{jd-supra_ip}

Ownership of IP varies depending on how a creation has been produced. If, for example, a developer creates a game independently, they generally retain full rights. In collaborative projects, or when work is commissioned, ownership can become complicated and must be defined in written contracts.\cite{silva_guide-to-release, jd-supra_ip}

When incorporating third-party source material such as characters, settings, or themes from movies, TV shows, or other media, licensing agreements must be secured. Even small references to copyrighted works can lead to legal action if not authorized.\cite{silva_guide-to-release, jd-supra_ip}

Beyond copyright, trademark protection can apply to titles, logos, and other branding elements. A trademark prevents competitors from using similar names or logos hence firstly ensuring that a game title and branding do not infringe on existing trademarks is crucial.\cite{dragon_copyright}

To legally include music in a game, two types of licenses can be obtained. A synchronization license grants the right to use the underlying composition whereas a master license grants the right to use a particular recording.\cite{iconcollective_music-license}

After the use of all assets has been approved an End-User License Agreement (EULA) needs to be drafted.\cite{silva_guide-to-release} It is used to set clear expectations and legally protects the interests of both the game developer and the player. It ensures that the creator retains ownership of its software, provides a framework for handling disagreements, limits the creator’s liability and ensures compliance with data privacy laws (like GDPR and CCPA). The agreement also allows users to understand what they're legally allowed to do with the software and provides specifications such as features and the functionality. Publishing platforms such as Steam provide a general EULA that usually covers the needs of a small game.\cite{docupilot_eula, steam_content-survey}

Developers must also ensure compliance with data privacy laws in the targeted market - the GDPR\cite{eu_gdpr} in Europe and the CCPA\cite{doj_ccpa} in the U.S. Data can generally only be collected if there is a lawful basis for it, a necessity for gameplay or user management and a clear explanation why and how it will be processed - usually in the EULA.\cite{dentons_eu-data-protection, gamota_data-privacy}

Traditionally, publishers required creators to obtain appropriate age ratings (e.g., PEGI, ESRB) based on the game’s content. On distribution platforms like Steam however, filling out a content survey suffices.\cite{steam_content-survey}

%---------------------------------------------------------------
\subsection{Marketing Requirements}
Marketing is the process of bringing a product to market. Successfully launching a game requires a strong marketing strategy to generate interest, attract customers, and maximize visibility.

During audience research, developers identify and attempt to understand the target audience. Different genres appeal to different player demographics, and marketing strategies should be tailored accordingly. Analyzing similar games, engaging with gaming communities or conducting surveys helps determine what resonates most.\cite{santos_pre-launch}

Creating a well-organized press kit is crucial for both media outreach and promotions. It should include high-quality trailers, screenshots, game descriptions, developer quotes, and release details.\cite{impress_press-kit}

Building an online presence is a common strategy indie developers employ to stand out, generate excitement and cultivate a dedicated community before launch. It is usually done through active participation - posting behind-the-scenes content, development updates, and engaging with fans - on social media platforms such as Reddit, Twitter, Instagram or TikTok. A website can serve as a central hub to direct an audience to. Its landing page should showcase press kit assets, include a newsletter signup option and other essential info.\cite{venkatesh_successful-release}

Influencer Partnerships - particularly with streamers and content creators aligned with the game’s genre - can significantly boost visibility.\cite{developer_introduction-to-marketing}

Finally, choosing the right distribution channel is key to a game’s launch and long-term success. Different platforms cater to different audiences, offer unique visibility opportunities, and have varying revenue-sharing models. Developers must evaluate their goals and pick the best fit:

\labelitemi \href{https://store.steampowered.com/}{Steam}\footnote{accessible through https://store.steampowered.com/} is the largest digital distribution platform for PC games, accounting for 50-70\% of global PC game downloads.\cite{zuckerman_steam-statistics} It offers powerful tools for developers, including community forums, game analytics, and built-in marketing features such as Steam Wishlists. Listing a game on Steam involves submitting it through Steamworks, paying a \$100 fee (refundable after \$1000 in sales), and adhering to the platform’s content guidelines. Steam has a fixed 30\% revenue split. The Steam Discovery Queue and algorithm-driven recommendations can boost sales provided the game gains enough initial traction through wishlists, reviews, and engagement.\cite{steam_wishlist,steam_partner-program,steam_discovery}

\labelitemi \href{https://itch.io/}{Itch.io}\footnote{accessible through https://itch.io/} is a flexible, developer-friendly platform known for its supportive indie community and experimental games. Unlike Steam, Itch.io allows the developer full control over the revenue split, offering a pay-what-you-want pricing model. Itch.io however lacks the built-in discovery mechanisms and massive audience of Steam, requiring developers to rely on external marketing.\cite{carpenter_creator-day}

\labelitemi \href{https://gamejolt.com/}{Game Jolt}\footnote{accessible through https://gamejolt.com/} emphasizes community-driven engagement with social-media-like features. Developers can post updates, interact with followers, and grow an audience over time. Game Jolt offers flexible monetization options, such as one-time purchases, donations, or ad-supported releases. While good for building a player base, Game Jolt too lacks the commercial reach of Steam.\cite{game-jolt_help}

For many indie developers, the best approach is releasing on multiple platforms - launch a free demo on Itch.io or Game Jolt before transitioning to a release on Steam.

%---------------------------------------------------------------
\subsection{Financial Requirements}
Understanding the financial requirements and strategies of game development determines the feasibility and success of a project. From budget planning, securing initial funding to monetization strategies, developers must navigate several financial challenges.

% - - - - - - - - - - - - - - - -
\paragraph{Budget-Planning} A well-structured budget - commonly created through a task breakdown\cite{linkedin_budget} - allocates funds across three primary areas:

\labelitemi \textbf{Development} includes salaries, outsourcing (e.g., music, voice acting) and tools (software/hardware).\cite{eduonix_costs}

\labelitemi \textbf{Marketing} covers promotional campaigns, ads, influencer partnerships, and events.\cite{eduonix_costs}

\labelitemi \textbf{Post-Launch Support} includes updates, fixes, server maintenance, and customer support.\cite{eduonix_costs}

% - - - - - - - - - - - - - - - -
\paragraph{Monetization Strategy} Choosing the right monetization strategy ensures sustainable business operations, allows investment into high-quality contributors and assets and incentivises innovation:

\labelitemi \textbf{Freemium} offers free access to the base game with revenue generated through ads or in-app purchases (e.g., skins).\cite{hubka_game-monetization}

\labelitemi \textbf{Premium} involves an upfront fee for the game and is sometimes supplemented by paid expansions.\cite{hubka_game-monetization}

\labelitemi \textbf{Subscription} provides access to the product throughout recurring payments (less common for indie games).\cite{hubka_game-monetization}

% - - - - - - - - - - - - - - - - 
\paragraph{Fundraising} Larger projects often require funding, which may come from several sources:

\labelitemi \textbf{Self-Funding} is commonly used by indie game developers until external funding is secured.\cite{perforce-stoftware_tips}

\labelitemi \textbf{Publishers} traditionally provide financial support and marketing expertise but may require revenue-sharing agreements.\cite{perforce-stoftware_tips}

\labelitemi \textbf{External investors} can financially back a project in exchange for equity or profit-sharing.\cite{perforce-stoftware_tips}

\labelitemi \textbf{Crowdfunding platforms} such as \href{https://gamefound.com/en}{Gamefound}\footnote{accessible through https://gamefound.com/en} or \href{https://www.kickstarter.com/}{Kickstarter}\footnote{accessible through https://www.kickstarter.com/} allow developers to raise funds directly from potential players but rely on strong promotional efforts.\cite{perforce-stoftware_tips}

\labelitemi \textbf{Organizations} in the gaming (eg. Unreal Engine) or education space may offer competitive grants to support prospective indie developers.\cite{perforce-stoftware_tips, unreal-engine_grants}

Lastly, when monetizing a game, the benefits of operating as a company should be considered. Starting a company is not strictly required and might entail upfront fees but provides legal protection, simplifies tax compliance, and can improve credibility when negotiating contracts with investors or publishers.\cite{expats_company-or-individual}