%===============================================================
\chapter{Student Venture Support Options}
%===============================================================

\begin{chapterabstract}
    While student game developers often excel at the creative and technical aspects of game creation, the transition from prototype to commercial product presents a host of new challenges - including legal and financial complexities. This chapter explores legal and organizational partnership formats for supporting student ventures, highlighting the limitations and emerging best practices.
\end{chapterabstract}

University incubators play a critical role in fostering student-led innovation - particularly in game development, where access to funding, mentorship, and infrastructure can make or break a start-up. Most university incubator programmes operate under the non-profit umbrella of their institutions. While this structure limits their ability to generate profit, it enables them to offer services such as mentoring, business guidance, co-working space, and financial assistance - often at no cost to students. Their primary goal is not immediate profit but rather fostering entrepreneurship and innovation.

Although some university incubators seek returns through royalties, equity ownership, or repayable loans, even large institutions often struggle to achieve financial self-sufficiency. As a result, these programmes are typically supported through a mix of institutional funding, public grants, corporate sponsorship, and alumni donations.

In many cases, incubators also manage the licensing of intellectual property created on university premises, including student projects - especially when the university retains certain rights, as is the case at FIT CTU. 

Commercializing student-developed games introduces legal and organizational complexities. Students must select suitable business structures and comply with tax and labor regulations. Incubators can simplify this process by incorporating ventures - protecting students from personal liability, facilitating profit-sharing and fair ownership.

%---------------------------------------------------------------
\section{Student Partnership Formats}
University incubators in the Czech Republic can support students in various ways, each offering unique advantages and constraints.

% - - - - - - - - - - - - -
\subsection{Supporting a Student-Led Venture}
\label{sec:label-supporting-student-venture}
One common method for universities to support student-led ventures is through structured partnerships, which allow students to operate independently while receiving institutional support. Common Legal Structures for Student Ventures include:
\begin{itemize}
    \item \textbf{Sole Proprietorship (OSVČ)} - ideal for ventures conducted individually. Requires no initial capital, can be registered by filing a unified registration form and paying an administration fee of CZK~1~000. The downside is full personal liability for any incurred debts and losses.
    \item \textbf{Limited Liability Company (s.r.o.)} - commonly used by small teams. Offers liability protection but requires structured book-keeping. Can be created with an initial capital of CZK~1 by concluding a memorandum of association (notary approval usually costs under CZK~10~000) and listing in the commercial register (administrative fees arround CZK~2~700 when done by a notary). 
    \item \textbf{Other - Joint-Stock Company (a.s.)} or Limited Partnership (komanditní společnost) may be appropriate for large projects seeking investment. Setting them up and adhering to the tax code is complex.
\end{itemize}

University incubators can support student-led venture through several financial mechanism:
\begin{itemize}
    \item \textbf{Loans} - repayable financing provided with agreed-upon terms and interest rates.
    \item \textbf{Equity Investment} - capital exchanged for partial ownership. Some non-profit structures cannot hold equity or engage in unlimited liability partnerships and need to have their business activity aligned with their defined purpose.
    \item \textbf{Grants} - non-repayable funding, often provided by non-profits. Can offer tax benefits to supporting organizations.
\end{itemize}

% - - - - - - - - - - - - -
\subsection{Employing Students}
University incubators could employ students directly or through subsidiary game development firms and compensate them for their contributions:
\begin{itemize}
    \item \textbf{Standard employment contracts} - developers receive a stable salary. Employers generally contribute 33.8\% (2.1\% towards sickness insurance, 21.5\% towards pension insurance, 1.2\% towards state employment policy and 9\% towards health insurance) while employees contribute an additional 6.5\% to social-security, 4.5\% towards health insurance and 15\% income tax. 
    \item \textbf{Freelance/contractor agreements}  - students work as independent contractors. They are required to register as sole proprietors, pay a 15\% income tax, and after deducting expenses pay 29.2\% social-security (if their yearly profits exceed CZK~111~736) and 13.5\% health insurance.
    \item \textbf{Internship programmes} - often tied to academic curriculum and enabled by scholarships funded by universities or industry partners. They must comply with Czech labor laws.
\end{itemize}

% - - - - - - - - - - - - -
\subsection{Allowing a Form of Ownership in the Incubator}
Students might also be allowed to hold equity in the incubator or its affiliated entities. This approach can minimize tax-burden and incentivize effort. Formats include:
\begin{itemize}
    \item \textbf{Paying out dividends} - distribution of profits to student shareholders. Such an approach requires the correct legal structure (in the Czech Republic usually an a.s., but can also be done through an s.r.o.) and has to be contractually defined. Transferring shares involves a CZK~2~000 administrative fee. Corporate profits are taxed 21\% and dividends an additional 15\%.
    \item \textbf{Profit distribution} - a cooperative (družstvo in the Czech Republic) allows members to participate in governance and share profits. At least 10\% of profits must be retained in a fund, with the rest distributed (subject to 21\% corporate and 15\% income tax). Initial costs include around CZK~3~000 for drafting statutes, CZK~10~000 for notarial certification and CZK~2~700 for listing in the commercial register.
\end{itemize}

%---------------------------------------------------------------
\section{University Incubator Legal Structure}
University incubators in the Czech Republic can take on various legal structures, each offering unique advantages and constraints.

% - - - - - - - - - - - - -
\subsection{Non-Profit Integration}
Many university incubators are structured as non-profit entities and integrated into the University's body. They are allowed to receive funding from public and private sources but focus on student development rather than profit-making. Common non-profit structures include:
\begin{itemize}
    \item \textbf{Student Clubs or Organizations} - informal collectives within universities usually offering peer-to-peer support and networking. Operations must strictly adhere to university policies and governance.
    \item \textbf{Joint Ventures with Private Companies} - often referred to as Public-Private Partnerships are not a legally described structure in the Czech Republic. Universities may partner with established game studios or tech firms to provide the required services.
    \item \textbf{Non-Profit Organizations} -  more formalized structures capable of applying for public grants and private donations. They are not allowed to be run for the sole purpose of generating profit.
    \begin{itemize}
        \item \textbf{Foundations (nadace)} - manage assets (exempt from tax) for charitable, religious, or public benefit goals. They are established by a notarial deed with an initial endowment (no minimal amount is specified) and by being listed in the foundations register. Endowments are protected and only earnings generated by its activities are expended. Foundation disclose annual financial reports, undergo audits and are restricted from participating in unlimited liability partnerships.
        \item \textbf{Funds (nadační fondy)} - provide access to the entire endowment. They focus on fundraising for specific causes or a time-bound project. Funds are registered by a notarial deed and by formulating governing statutes. They require no minimum capital, undergo lighter oversight but still disclose their activities.
        \item \textbf{Registered Institutes (ústavy)} - conduct educational, scientific, or cultural activities. They are established by concluding a memorandum of association, registering in the commercial register and have to undergo annual audits if revenues exceed CZK~40~million. Entrepreneurial activities must be supportive in pursuing the institute’s purpose.
    \end{itemize}
\end{itemize}

% - - - - - - - - - - - - -
\subsection{University-Owned Spin-Off Company}
Some universities establish separate, fully or partially owned entities. These must distribute gains to maintain the university’s nonprofit status. Common formats include:
\begin{itemize}
    \item \textbf{S.R.O. (Limited Liability Company)} - a university-owned company that partners with student teams created following the points mentioned in [chapter before].
    \item \textbf{A.S. (Joint-Stock Company)} - suitable for large ventures seeking external investment, requires initial capital of at least 2 million CZK.
    \item \textbf{Cooperative Structure }- a collective ownership model where students, faculty, and external partners share decision-making and profits. A cooperative (družstvo in the Czech Republic) is a corporate entity requiring a minimum of 3 members. It is founded by concluding a memorandum of association (notary approval usually costs under CZK 10000) and by listing the company in the commercial register (notary fee usually around CZK 2700). Profit distribution rules need to be outlined in the cooperative’s memorandum and can include criteria beyond capital contributions  - such as the performance of specific segments of the organization. Distributed profits are subject to a 21% corporate tax and a 15% personal income tax.
\end{itemize}
