%===============================================================
\chapter{Game Development Processes}
%===============================================================

\begin{chapterabstract}
	Lorem ipsum dolor sit amet, consectetuer adipiscing elit. Curabitur sagittis hendrerit ante. Class aptent taciti sociosqu ad litora torquent per conubia nostra, per inceptos hymenaeos. Cras pede libero, dapibus nec, pretium sit amet, tempor quis. Sed vel lectus. Donec odio tempus molestie, porttitor ut, iaculis quis, sem. Cras pede libero, dapibus nec, pretium sit amet, tempor quis. Sed vel lectus. 
\end{chapterabstract}

Lorem ipsum dolor sit amet, consectetuer adipiscing elit. Curabitur sagittis hendrerit ante. Class aptent taciti sociosqu ad litora torquent per conubia nostra, per inceptos hymenaeos. Cras pede libero, dapibus nec, pretium sit amet, tempor quis. Sed vel lectus. Donec odio tempus molestie, porttitor ut, iaculis quis, sem. Suspendisse sagittis ultrices augue. Donec ipsum massa, ullamcorper in, auctor et, scelerisque sed, est. In sem justo, commodo ut, suscipit at, pharetra vitae, orci. Pellentesque pretium lectus id turpis.

%===============================================================
\section{Game Development Stages}
Game development is the process of designing, creating, and releasing video games. It includes writing, sound design, project management, programming and more. The process can be divided into distinct stages that focus on different aspects of the final product.

%---------------------------------------------------------------
\paragraph{Planning Stage}
At this stage, game developers choose the genre that is suitable for their ideas, select viable art styles and gameplay mechanics, plan the game’s structure, content, and more. Changing, cutting or replacing some aspects of the game later on can be easy, where others can be challenging and must be decided in the early stages.

%---------------------------------------------------------------
\paragraph{Pre-Production Stage}
The pre-production stage of game development requires artists, writers and designers to finalise important decisions. Feasibility, practicality and the worth of different design aspects is considered. Will the game be fun to play and appealing to look at? Will it work properly, or do some technical limitations need to be taken into account?

%---------------------------------------------------------------
\paragraph{Production Stage}
After the decision-making, production of the game can start. It is at this stage when most of the code is written, levels are designed, game mechanics are tested, models, textures and visual elements start to appear.

%---------------------------------------------------------------
\paragraph{Testing Stages}
Some form of internal testing is done throughout the entire process. Before the game is finalised however, developers tend to release test versions. This practice can be roughly divided into alpha and beta.

The alpha version of the game already has the key mechanics and allows developers to assess playability. It might have placeholders for characters, surroundings or lack music. It is used for internal - closed - testing between staff members but can in some cases be available - open - to selected, passionate fans willing to help developers with playtesting.

The beta version follows alpha. The game still requires a lot of work at this point but this is where areas such as the environment and characters are taking its final form. There still might be bugs present, glitches and exploits that need fixing, performance optimization required, and details missing. The game mechanics may still need to be balanced and server stability tested. Betas can too be open or closed.

%---------------------------------------------------------------
\paragraph{Launch Stage}
During the launch stage the game is published for the public to play. This stage requires understanding the target market, audience, selecting a distribution channel, creating a strategy and promoting. Additional support for players might be provided and feedback gathered.

%---------------------------------------------------------------
\paragraph{Post-Launch Stage}
After the initial publication, developers might want to release updates, patch bugs or even add new content, either as a free update or in the form of a purchasable extension. Continuation of a successful product allows it to extend its lifespan and provides a long-term fanbase.

%===============================================================
\section{Launch Process In Detail}
While student developers often excel at designing and programming games, the launch phase is where many projects struggle. Unlike development, which follows a structured technical process, launching a game involves a complex and often unfamiliar set of tasks, from marketing and distribution to niche legal and financial specialties, budget planning, fundraising, assessing copyright protection, trademarks, creating contracts and more. A successful launch requires careful planning, strategic timing, and an understanding of distribution platforms and promoting. Many of these steps are not immediately obvious but can determine whether a game finds an audience or gets lost in an oversaturated market. This section further breaks down the critical components of a game launch.

%---------------------------------------------------------------
\subsection{Structural Requirements}
Ensuring a smooth and successful launch requires meeting critical structural requirements that impact a game’s performance, security, and compliance. Failing to address these factors can lead to negative user experiences, security vulnerabilities, and even regulatory consequences.

Before a game is released to the public, extensive testing is employed to ensure stability and playability. This process is typically conducted in several stages including alpha testing and beta testing which help refine the game and minimize post-launch patches.

The first thing a player interacts with in a game however is the UI - a launch screen or app - which therefore needs to be optimized. Settings such as the resolution, window size, language, subtitles and key bindings need to work properly. Accessibility and support options need to be tested and credits/end game screens polished.

For games with online components, a reliable server infrastructure is crucial. Poor server performance can lead to lag or disconnects during traffic surges. Optimizing server configuration, considering scalability and running stress tests before launch helps identify potential bottlenecks.

Major gaming platforms, from Steam to PlayStation, Xbox and mobile app stores, have specific technical requirements. Failing to meet requirements such as performance specifications or file size limitations can lead to rejection, or post-launch issues.

Collecting and storing personal player data is best avoided in the case of small student-led projects. Developers must comply with regulations in the target regions such as the General Data Protection Regulation (GDPR) in Europe and the California Consumer Privacy Act (CCPA) in the U.S. These regulations require data to be anonymized, encrypted, safely stored and access to it minimized. Non-compliance can lead to severe penalties.

%---------------------------------------------------------------
\subsection{Operational Requirements}
A successful game launch requires careful coordination of tasks and resources. While technical readiness is crucial, the operational aspects of the launch determine how smoothly the transition from development to release unfolds.

A good practice is creating a detailed timeline for launch-related activities and setting a realistic launch date. By mapping out the critical milestones, teams can avoid last-minute chaos.

Seamless collaboration between developers, marketering, community management, and support staff is required for launch. To prevent miscommunication or overlooked tasks, teams should clearly define individual responsibilities and establish a contingency plan in case of unexpected problems. A pre-launch meeting can ensure that all team-members are aligned and ready for launch day challenges.

Preparing announcements across player communication channels (e.g., Discord, Reddit) to address potential issues or provide updates ensures support teams are ready to handle player inquiries promptly.

%---------------------------------------------------------------
\subsection{Legal Requirements}
From a legal standpoint, publishing a game requires ensuring compliance with intellectual property laws, consumer rights, and distribution agreements.

Intellectual Property (IP) Protection guards creators’ work with copy rights, trademarks, and patents. IP protection applies to a finished game, but might also restrict use of assets such as code, art, music, branding.

Copyright ownership varies depending on how a game is developed. If a developer creates a game independently, they generally gain full rights to their work. In cases where multiple people contribute, or when work is commissioned, ownership rights can become complex and should be underscored in written agreements.

When incorporating third-party source material such as characters, settings, or themes from movies, TV shows, or other media, licensing agreements must be secured. Even small references to copyrighted works can lead to legal action if not authorized.

Beyond copyright, trademark protection can apply to titles, logos, and other branding elements. A trademark prevents competitors from using similar names or logos hence firstly ensuring that a game title and branding do not infringe on existing trademarks is crucial.

To be allowed to include music in a game, two different types of licenses are possible. A synchronization license grants the right to use the underlying composition whereas a master license grants the right to use a particular recording.

An End-User License Agreement (EULA) sets clear expectations and legally protects the interests of both the game developer and the player. It ensures that the creator retains ownership of its software, provides a framework for handling disagreements, limits the creator’s liability and ensures compliance with data privacy laws (like GDPR and CCPA). The agreement also allows users to understand what they're legally allowed to do with the software and provides specifications such as features and the functionality. Publishing platforms such as Steam provide a general EULA that usually covers the needs of a small game. 

Developers must comply with data privacy laws in the targeted market - the General Data Protection Regulation (GDPR) in Europe and the California Consumer Privacy Act (CCPA) in the U.S. Data can generally only be collected if there is a lawful basis for it, a necessity for gameplay or user management and a clear explanation why and how it will be processed - usually in the EULA.

Traditionally, publishers required creators to obtain appropriate age ratings (e.g., PEGI, ESRB) based on the work’s content. On distribution platforms like Steam, filling out a content survey suffices.

%---------------------------------------------------------------
\subsection{Marketing Requirements}
Marketing is the process of bringing a product to market. Successfully launching a game requires a strong marketing strategy to generate interest, attract players, and maximize visibility. To ensure a successful release, developers must consider marketing components from audience research to online presence and distribution platforms.

In audience research developers identify and attempt to understand the target audience. Different genres appeal to different player demographics, and marketing strategies should be tailored accordingly. Analyzing similar games, engaging with gaming communities or conducting surveys helps determine what resonates most.

Creating a well-organized press kit is crucial for media outreach. It should include high-quality trailers, screenshots, game descriptions, developer quotes, and release details to make it easy for journalists and content creators to cover the game.

Building an online presence is a common strategy indie developers employ to stand out, generate excitement and cultivate a dedicated community before launch. It is usually done through active participation - posting behind-the-scenes content, development updates, and engaging with fans - on social media platforms such as Reddit, Twitter, Instagram or TikTok. A website can serve as a central hub to direct an audience to. A well-designed website landing page should include features from a press kit - high-quality trailers, screenshots, a compelling game description, release details - and a newsletter signup option.

Partnerships with influencers and streamers are a fundamental part of modern game marketing. Collaborating with content creators who align with the game’s genre and audience can significantly increase visibility.

Finally - choosing the right distribution channel and publishing the game - the prominent steps of the game launch and the game development process generally. Different platforms cater to different audiences, offer unique visibility opportunities, and have varying revenue-sharing models. Developers must assess their options to determine which platform best aligns with their audience and business goals.

Steam - the industry giant - is the largest and most influential digital distribution platform for PC games, accounting for 50-70% of global PC game downloads. It offers powerful tools for developers, including community forums, game analytics, and built-in marketing features such as Steam Wishlists. Listing a game on Steam requires submitting it through Steamworks, paying a $100 listing fee (which can be returned after a $1000 generated in sales), and adhering to the platform’s content guidelines. Steam has a fixed 30% revenue split. The Steam Discovery Queue and algorithm-driven recommendations can boost sales provided the game gains enough initial traction through wishlists, reviews, and engagement.

Itch.io is a more flexible, developer-friendly distribution platform. It is known for its supportive indie community and experimental games. Unlike Steam, Itch.io allows the developer full control over the revenue split. Its store pages can be extensively customized, offer pay-what-you-want pricing models. However, Itch.io lacks the built-in discovery mechanisms and massive audience of Steam, requiring developers to entirely market their games through community engagement, social media and influencer partnerships.

Game Jolt is a distribution platform that focuses on community-driven engagement and social features. Unlike Steam, Game Jolt provides social media-like functions allowing developers to post updates, interact with followers, and grow an audience over time. Game Jolt too offers developer-friendly monetization options, allowing the sale of games as one-time purchases, donations, or ad-supported releases. This makes it a good choice for player base building, however just like Itch.io, Game Jolt lacks the commercial reach of Steam.

For many indie developers, the best approach is releasing on multiple platforms - launch a free demo on Itch.io or Game Jolt before transitioning to a release on Steam.

%---------------------------------------------------------------
\subsection{Financial Requirements}
Understanding the financial requirements and strategies of game development determines the feasibility and success of a project. From budget planning, securing initial funding to monetization strategies, structural considerations and managing post-launch revenue, developers must navigate several financial challenges.

A well-structured budget allocates funds across three primary areas. Development includes salaries for developers, artists, and designers, outsourced tasks such as music composition and voice acting and costs for software and hardware. Marketing covers promotional campaigns, ads, influencer partnerships, and events. Post-Launch Support includes updates, bug fixes, server maintenance for online games, and customer support. A detailed task breakdown can help estimate costs more accurately.

Choosing the right monetization strategy ensures sustainable business operations, allows investment into high-quality contributors and assets, incentivises innovation and can support free-to-play models. Freemium offers free access to the base game with revenue generated through ads or in-app purchases (e.g., skins). The Premium model charges an upfront fee for the game and is sometimes supplemented by paid expansions. The - in games less common - Subscription model provides access to the product throughout the duration of recurring payments of a fee. 

Securing funding is often necessary for larger projects but can be challenging. Self-Funding is a common initial investment source for indie game developers. Founders use personal savings until external funding is secured. Traditionally, publishers provide financial support and marketing expertise but may require revenue-sharing agreements. External investors can financially back a project in exchange for equity or profit-sharing. Crowdfunding platforms such as Gamefound or Kickstarter allow developers to raise funds directly from potential players but rely on strong promotional efforts. Organizations - in the gaming (eg. Unreal Engine) or education space - may offer competitive grants to support up and coming indie developers.

Lastly, when monetizing a game, the benefits of operating as a company should be considered. Starting a company is not strictly required and might entail upfront fees but provides legal protection, simplifies tax compliance, and enhances credibility when negotiating contracts with investors or publishers. Operating as an individual may also limit opportunities for partnerships with publishers or platforms.
