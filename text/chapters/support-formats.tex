\externaldocument[nocite]{build/text/chapters/incubators}[url=text/chapters/incubators]
%===============================================================
\chapter{Formats of Student Entrepreneurship Support}\label{chap:support-formats}
%===============================================================

\begin{chapterabstract}
    While student game developers often excel at the creative and technical aspects of game creation, the transition from prototype to commercial product presents a host of new challenges---including legal and financial complexities. This chapter explores legal and organisational partnership formats for~supporting student ventures, highlighting the limitations and emerging best practices.
\end{chapterabstract}

University incubators—as described in chapter \ref{chap:incubators}---play a critical role in fostering student-led innovation---particularly in game development, where access to~funding, mentorship, and infrastructure can make or break a start-up.

Most university incubator programmes operate under the non-profit umbrella of their institutions. While this structure limits their ability to generate profit, it enables them to offer services such as mentoring, business guidance, co-working space, and financial assistance---often at no cost to students.
\cite{uni-game-incubation, isis-innovation}

However, students and university staff frequently explore alternative formats, particularly when institutional frameworks prove too slow, or limited in scope. To better understand the practical options available to students navigating the~transition to market entry and to institutions providing support, it is important to explore a range of models.
\cite{uni-game-incubation, isis-innovation}

%---------------------------------------------------------------
\section{Partnership Formats}\label{sec:partnership-formats}
University incubators in the Czech Republic can support students in various ways, each offering unique advantages and constraints.

% - - - - - - - - - - - - -
\paragraph{Supporting a Student-Led Venture}\label{subsec:supporting-student-venture}
One common method for universities to support student-led ventures is through structured partnerships. These arrangements allow students to operate independently while receiving institutional support through different financial mechanisms \cite{fundinvoice-funding}:
\begin{itemize}
    \item \textbf{Loans:} Repayable financing provided with agreed-upon terms and interest rates.
    \item \textbf{Equity Investment:} Capital exchanged for partial ownership. Some non-profit structures cannot hold equity or engage in unlimited liability partnerships.
    \item \textbf{Grants:} Non-repayable funding, often provided by non-profits. Can offer tax benefits to supporting organisations.
\end{itemize}
The student venture receiving such support has to have a legal structure, such~as:
\begin{itemize}
    \item \textbf{Sole Proprietorship (OSVČ):} Ideal for ventures conducted individually. Requires no initial capital, can be registered by filing a unified registration form and paying an administration fee of CZK~1~000. The downside is full personal liability for any incurred debts and losses. \cite{fakturoid-osvc, Zapletalova-osvc, lano-payroll, mpsv-social-sec}
    \item \textbf{Limited Liability Company (s.r.o.):} Commonly used by small teams. Offers liability protection but requires structured book-keeping. Can be created with an initial capital of 1~CZK by signing a memorandum of~association (notary approval usually costs under CZK~10~000) and listing in the commercial register (administrative fees around CZK~2~700 when done by a notary). \cite{vajda-llc, lano-payroll, mpsv-social-sec}
    \item \textbf{Other:} A Joint-Stock Company (a.s.) or a Limited Partnership (komanditní společnost) may be appropriate for large projects seeking investment. Setting them up and adhering to the tax code is complex.\cite{jake-obchodni-spolecnosti}
\end{itemize}

% - - - - - - - - - - - - -
\paragraph{Employing Students}
University incubators could also employ students—directly or through subsidiary game development firms—and compensate them for their contributions:
\begin{itemize}
    \item \textbf{Standard employment contracts:} Developers receive a stable salary. Employers contribute 33.8\% (2.1\% towards sickness insurance, 21.5\% towards pension insurance, 1.2\% towards state employment policy and 9\% towards health insurance) while employees pay an additional 6.5\% to social-security, 4.5\% towards health insurance and 15\% income tax. \cite{lano-payroll, mpsv-social-sec}
    \item \textbf{Freelance/contractor agreements:} Students work as independent contractors. They are required to register as sole proprietors, pay a 15\% income tax, and—once their yearly profits exceed CZK~111~736—pay 29.2\% social-security and 13.5\% health insurance from income after deducting expenses. \cite{fakturoid-osvc, Zapletalova-osvc, lano-payroll, mpsv-social-sec}
    \item \textbf{Internship programmes:} Often tied to academic curriculum and enabled by scholarships funded by universities or industry partners. They must comply with Czech labour laws.
\end{itemize}

% - - - - - - - - - - - - -
\paragraph{Allowing a Form of Ownership in the Incubator}
Students can also be allowed to hold equity in the incubator or its affiliated entities—such as a game studio—and profits then distributed to them. This approach can minimize tax-burden and incentivize effort. Such an approach requires the correct legal structure (in the Czech Republic usually a s.r.o., a.s., or a cooperative) and has to be contractually defined. Transferring ownership of an s.r.o. involves a CZK~2~000 administrative fee, whereas cooperatives offer a simple transfer mechanism. Corporate profits are taxed 21\% and dividends\footnote{Dividends are a sum of money paid regularly to a company’s shareholders out of its profits (or reserves).} an additional 15\%.
\cite{vajda-llc}

%---------------------------------------------------------------
\section{Incubator Legal Structures}
University incubators in the Czech Republic can take on various legal structures, each offering unique advantages and constraints.

% - - - - - - - - - - - - -
\paragraph{Non-Profit Organisation}
Many university incubators are structured as~non-profit entities. They are usually integrated into the university's body, but with enough funding could be established separately. They are allowed to receive funding from public and private sources but focus on student development rather than profit-making. Common non-profit structures include:
\begin{itemize}
    \item \textbf{Student Clubs or Organisations:} Informal, non-commercial collectives---usually within universities---offering peer-to-peer support and networking. \cite{lily-NONPROFIT}
    \item \textbf{Joint Ventures with Private Companies:} Often referred to as Public-Private Partnerships are not a legally described structure in the Czech Republic. Universities and staff may partner with established game studios or tech firms to provide the required services. \cite{lily-NONPROFIT}
    \item \textbf{Non-Profit Organisations:} More formalized structures capable of applying for public grants and private donations. They are not allowed to be run for the sole purpose of generating profit. \cite{lily-NONPROFIT}
    \begin{itemize}
        \item \textbf{Foundations (Nadace):} manage assets (exempt from tax) for charitable, religious, or public benefit goals. They are established by a notarial deed with an initial endowment (no minimal amount is specified) and by being listed in the foundations register. Endowments are protected and only earnings generated by its activities are expended. Foundation disclose annual financial reports, undergo audits and are restricted from participating in unlimited liability partnerships.
        \item \textbf{Funds (Nadační fondy):} Provide access to the entire endowment. They focus on fundraising for specific causes or a time-bound project. Funds are registered by a notarial deed and by formulating governing statutes. They require no minimum capital, undergo lighter oversight but still disclose their activities.
        \item \textbf{Registered Institutes (Ústavy):} Conduct educational, scientific, or~cultural activities. They are established by concluding a memorandum of association, registering in the commercial register and have to undergo annual audits if revenues exceed CZK~40~million. Entrepreneurial activities must be supportive in pursuing the institute’s purpose.
    \end{itemize}
\end{itemize}

% - - - - - - - - - - - - -
\paragraph{University-Based Spin-Off Companies}
University-affiliated staff commonly establish separate entities that may operate for profit. Common formats include:
\begin{itemize}
    \item \textbf{S.R.O. (Limited Liability Company):} A university-owned company that partners with student teams created following the points mentioned in section \ref{sec:partnership-formats}.
    \item \textbf{A.S. (Joint-Stock Company):} Suitable for large ventures seeking external investment, requires initial capital of at least CZK~2~million. \cite{jake-obchodni-spolecnosti}
    \item \textbf{Cooperative Structure:} A collective ownership model where students, faculty, and external partners share decision-making and profits. A cooperative (družstvo in the Czech Republic) is a corporate entity requiring a~minimum of 3 members. It is founded by concluding a memorandum of~association. Profit distribution rules need to be outlined in the cooperative’s statutes and can include criteria beyond capital contributions---such as the performance of specific segments of the organisation. Dividends are subject to a 21\% corporate tax and a 15\% personal income tax. \cite{preuss-coop, team-coop, businessinfo-coop}
\end{itemize}
