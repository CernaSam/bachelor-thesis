%===============================================================
\chapter{Academic Entrepreneurial Ventures}
%===============================================================

\begin{chapterabstract}
	Lorem ipsum dolor sit amet, consectetuer adipiscing elit. Curabitur sagittis hendrerit ante. Class aptent taciti sociosqu ad litora torquent per conubia nostra, per inceptos hymenaeos. Cras pede libero, dapibus nec, pretium sit amet, tempor quis. Sed vel lectus. Donec odio tempus molestie, porttitor ut, iaculis quis, sem. Cras pede libero, dapibus nec, pretium sit amet, tempor quis. Sed vel lectus. 
\end{chapterabstract}

Lorem ipsum dolor sit amet, consectetuer adipiscing elit. Curabitur sagittis hendrerit ante. Class aptent taciti sociosqu ad litora torquent per conubia nostra, per inceptos hymenaeos. Cras pede libero, dapibus nec, pretium sit amet, tempor quis. Sed vel lectus. Donec odio tempus molestie, porttitor ut, iaculis quis, sem. Suspendisse sagittis ultrices augue. Donec ipsum massa, ullamcorper in, auctor et, scelerisque sed, est. In sem justo, commodo ut, suscipit at, pharetra vitae, orci. Pellentesque pretium lectus id turpis.

%===============================================================
\section{Start-Ups and Spin-Outs}
A start-up is a business newly established by an entrepreneur that aims to develop a unique product or service to meet market needs. Startups are characterized by innovation, high growth potential and often low revenue. They typically operate in uncertain environments, relying on venture capital, angel investors, or other funding sources to support their development.

A spin-out is a new company created from an existing organization, such as a university or a corporation, formed to commercialize developed research, intellectual property, or technology. Spin-outs benefit from the resources and expertise of their parent organizations while functioning as independent entities.

In academia, startups and spin-offs translate ideas and research into marketable products and services. Universities and research institutions support these ventures through incubators, technology transfer offices, and funding programs. Academic spin-outs, in particular, utilize faculty expertise, patents, and support to drive innovation and economic impact.

%===============================================================
\section{The Impact of Academia-Driven Innovation}
Student spin-outs transfer theoretical early-stage research from universities to the market. They are seen as an attractive alternative to patent licensing as they are more likely to impact the local economy. Spin-outs create jobs for highly skilled workers and provide valuable knowledge spillover for other companies.

Students taking part in spin-outs can gain experience in entrepreneurship, practice business planning, market analysis, technology commercialization and more. Even failed ventures provide valuable learning opportunities. Moreover, spin-outs offer an additional career path for students (particularly those facing limited academic job opportunities).

Spin-outs foster collaboration between students, faculty, and external actors such as industry partners. This strengthens the university's ecosystem for innovation and entrepreneurship. Successful spin-outs improve the university's reputation and potentially attract more funding for research and other programmes.

Academic spin-outs have demonstrated the ability to adapt quickly to societal challenges. For example, during the COVID-19 pandemic, some quickly developed solutions to address urgent problems. Many of them also focus on solving pressing global problems, such as sustainability or healthcare challenges, contributing to broader societal benefits.

Spin-outs offer many benefits, from driving local economic growth to fostering innovation and creating new career opportunities for students. Supporting these ventures is a logical step, as is encouraging the commercialization of student-developed games. Allowing finished games and projects to be forgotten is a wasted opportunity. By supporting these creations, universities can increase their impact, accelerate students' careers and strengthen the entrepreneurial ecosystem.

%===============================================================
\section{Academia-Driven Innovation in the Czech Republic and the World}
The commercialization and success rates of student ventures vary significantly across the world, reflecting differences in ecosystems, funding structures, and cultural attitudes toward entrepreneurship.

In a study conducted by the Research, Development and Innovation Council in 2021, the Czech Republic demonstrated strong economic potential and a well-established industrial and research base. Investment in research and development reached a record CZK 121.9 billion (2\% of GDP). The country showed a high publication output, with over 80\% of research results published in indexed journals. Increasing international collaboration has driven excellence in specific scientific fields.

The study however concluded that challenges remain. The PhD completion rates are declining, reflecting low development of researchers' skills and professional capacities. Cooperation between the private and public sectors in research, development and innovation is limited. Innovation faces hurdles such as funding shortages and administrative burdens.

The Czech students perceive entrepreneurship as an attractive option for their future career paths. Addressing the systematic weaknesses while capitalising on the existing strengths can improve the Czech Republic’s competitiveness in research, development, and innovation on the global stage.

The gaming industry produces the Czech Republic’s most significant cultural export according to infiniteczechgames.com. Data by the Czech Game Developers Association from 2023 attributed the industry a turnover of CZK 7.52 billion with more than 98\% of it from abroad.

Mostly comprised of small studios, the industry employed over 2600 workers in 2023 - doubling since 2007. 71\% of those workers were under the age of 35 but only 48\% university-educated and only 21.4\% stated formal education as a source of their know-how. Nearly 90\% of all Czech game production is distributed and promoted without the support of publishers and distributors. These numbers suggest that the field is vibrant and growing, but that the impact of universities and university-provided support is limited.
