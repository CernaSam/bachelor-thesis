\externaldocument[nocite]{build/text/chapters/game-dev}[url=text/chapters/game-dev]
\externaldocument[nocite]{build/text/chapters/support-formats}[url=text/chapters/support-formats]
\externaldocument[nocite]{build/text/chapters/launch-&-testing}[url=text/chapters/launch-&-testing]
\externaldocument[nocite]{build/text/chapters/programme}[url=text/chapters/programme]
\externaldocument[nocite]{build/text/appendix}[url=text/appendix]
%===============================================================
\chapter{Legal Structure for Our Programme}\label{chap:legal}
%===============================================================

\begin{chapterabstract}	
    Student-led game development is inherently creative and unpredictable, often lacking the business structure needed to succeed in a competitive market. Grafit.games can provide that structure. This chapter evaluates various legal models for university-affiliated incubators, ultimately proposing a phased approach beginning with informal student clubs and transitioning into a cooperative. The cooperative model has been chosen for its democratic governance, shared ownership, ability to fairly distribute profits among members, and low bureaucracy. It incorporates informal teams, providing a legal shell. The chapter provides statutes—describing membership, governance, and capital contributions---and guides creation of the entity. Additionally it describes the handling of Intellectual property rights using non-exclusive licences---allowing the incubator to commercialise games and student developers to retain rights to their work.
\end{chapterabstract}

Transitioning a prototype to a market-ready product is an exciting yet formidable journey. Providing mentorship and consulting goes a long way in facilitating that mission. Game development ventures—particularly at the student level—are unpredictable by nature---they rely on creative momentum in a competitive market. As such, navigating the legal, financial, and administrative landscape can pose an insurmountable obstacle. In section \ref{sec:swot}, it led to the conclusion that business support should be a prominent feature of the programme. While such services can begin informally, with growing complexity establishing a~formal legal structure becomes necessary. Facilitating consistency, effective management of responsibilities, and ensuring long-term sustainability, the selection of an appropriate legal structure is critical.

%===============================================================
\section{Selecting a Legal Structure}
In chapter~\ref{chap:support-formats}, I outlined the various possible formats of student venture support and the legal structures university incubators may take in the Czech Republic. Each offers distinct benefits and constraints.

%- - - - - - - - - - - - - - - - - - - - - - - - - - - - - - - -
\paragraph{Partnership Formats}
While \textbf{financial support} is a valuable feature of~a~support programme, evidence from other university incubators suggests that equity- or royalty-based models are rarely sustainable. As such, Grafit.games will offer monetary support primarily in the later stages of the project and, wherever possible, free of charge.

Directly \textbf{employing students} is not aligned with the programme’s objectives. Czech labour laws require that students be financially compensated for work performed. Without a sustainable revenue model, such a framework becomes infeasible.

Instead, the programme focuses on offering mechanisms for \textbf{student ownership} within the incubator. This approach allows for centralized administrative and legal responsibilities, reducing the burden on student developers. By incorporating student games under a unified structure, the incubator can manage accounting, legal compliance, and profit distribution on students’ behalf.

%- - - - - - - - - - - - - - - - - - - - - - - - - - - - - - - -
\paragraph{Legal Structures—Non-Profit}
Non-profit models allow integration into the university’s non-profit structure but can also be created independently. However, because they are legally prohibited from distributing profits to students, they do not align with the programme’s long-term entrepreneurial objectives. Common examples include student clubs, public-private partnerships, and non-profit organisations.

\textbf{Student Clubs} are informal, low-cost collectives of students and staff. They provide a mechanism useful for testing the educational model. They, however, lack the legal and commercial tools necessary for supporting ventures beyond an academic context. Business support would remain entirely the students’ responsibility.

\textbf{Public-Private Partnerships} could involve university educators mentoring students while private studios handle publishing. While valuable for high-potential projects, these collaborations are often difficult to secure and benefit only a small subset of participants. They are unlikely to offer broad commercial support for most student ventures.

\textbf{Non-Profit Organisations} benefit from grant eligibility and tax exemptions but are subject to intensive regulation, requiring complex administration and robust bookkeeping. They cannot directly distribute profits to students. While not ideal for the~early commercial phases, a non-profit may become a suitable option once the educational mechanism is validated and sufficient partnerships and funding are secured.

%- - - - - - - - - - - - - - - - - - - - - - - - - - - - - - - -
\paragraph{Legal Structures—Spin-Off Company}
Forming a for-profit spin-off company will allow student ownership and profit distribution. Different models offer varying levels of flexibility and scalability.

\textbf{A joint-stock company} is well-suited for attracting investors, scaling large ventures, and provides strong mechanisms for profit distribution. However, it~requires at least CZK 2 million in initial capital. It is more appropriate for the later stages of the incubator and large-scale initiatives.

\textbf{A limited liability company} has low startup costs and a straightforward method for distributing profits via dividends to owners—avoiding the need for~social or health insurance contributions. It might be ideal for student-founded start-ups, but since ownership changes require recording in the commercial register (subject to CZK 2 000 fee), it does not easily allow incorporation of students. This model is therefore solely an extension of the student club structure.

\textbf{A cooperative (družstvo)} allows students, faculty, and partners to share ownership, profits, and decision-making. Though administratively more complex than an s.r.o., it suits the long-term goals well.

%- - - - - - - - - - - - - - - - - - - - - - - - - - - - - - - -
\paragraph{Priorities of Grafit.games}
To balance simplicity and potential, I have selected a phased approach:

\begin{enumerate}
    \item \textbf{Initial Phase:} Begin with an informal student club to pilot the educational portion of the programme. This low-cost model allows for rapid testing and adaptation without formal legal obligations.
    \item \textbf{Growth Phase:} Once demand for business support emerges, formalize operations using a cooperative structure—for collaborative ventures where shared ownership and reduced administrative burden are prioritised.
    \item \textbf{Future Phase:} Once the incubator attracts significant partners, investors, or establishes a large fund, it can transition into a Joint-Stock Company, Fund, or Foundation, depending on its evolved purpose and scale.
\end{enumerate}

This step-by-step approach ensures that the mechanism adapts and is legally sound. Preparing the cooperative structure in advance is essential. Even though I envision three distinct phases of growth, the project must be ready to implement the structure seamlessly as soon as it's needed.

%===============================================================
\section{Defining the Cooperative}\label{sec:defining-coop}
Setting up a cooperative allows democratic governance, equitable profit sharing, and joint control of creative and financial direction. By implementing the~laid-out structure, a collective of developers can formalize collaboration without sacrificing autonomy or flexibility. The following chapter outlines statutes (found in appendix \ref{ap:statutes}) created for Grafit.games, inspired by similar existing documents\cite{drevojas, stanovy-brno}, and guided by Act No. 90/2012 Coll., on Commercial Corporations\cite{ZOK}.

%- - - - - - - - - - - - - - - - - - - - - - - - - - - - - - - -
\paragraph{Legal Form and Purpose}
The cooperative must be established as a \czquote{výrobní družstvo} (production cooperative), a type of business corporation under the~Czech Act on Business Corporations. It is open-ended in terms of number of~members and serves both a business purpose and the common interests of~its members.

\begin{quote}
    Article I, Section 4: \czquote{Družstvo je společenstvím neuzavřeného počtu osob založeným za účelem podnikání. Družstvo je obchodní korporací podle zákona o obchodních korporacích.}
    
    (\enquote{A cooperative is a community of an unlimited number of persons established for the purpose of conducting business. It is a business corporation under the Act on Business Corporations.})
\end{quote}

The declared scope of activity should include:
\begin{itemize}
    \item Game development (including related services like QA, graphics, music, writing, publishing, events),
    \item Marketing and media services in digital entertainment,
    \item R\&D in digital technologies and media.
\end{itemize}

%- - - - - - - - - - - - - - - - - - - - - - - - - - - - - - - -
\paragraph{Membership}
Membership is open to:
\begin{itemize}
    \item individuals over 18 years old,
    \item teams (at least 2 individuals with an internal agreement),
    \item legal entities (companies, organisations).
\end{itemize}
An applicant must submit a simple written application. Membership is granted by decision of the cooperative’s chairman. Application must include:
\begin{itemize}
    \item name and address (for individuals),
    \item team name and addresses of members (for teams of individuals),
    \item legal name and registered office (for legal bodies),
    \item address (digital) for official communication.
\end{itemize}

\begin{quote}
    Article III, Section 4: \czquote{Tým je v družstvu zastoupen jedním statutárním zástupcem, který vykonává práva a povinnosti člena družstva za celý tým.}

    (\enquote{A team is represented in the cooperative by one statutory representative, who exercises the rights and responsibilities of a cooperative member on behalf of the entire team.})
\end{quote}
\begin{quote}
    Article III, Section 5: \czquote{Vstup jednotlivých fyzických osob do týmu a~jejich vystoupení z týmu se řídí vnitřní dohodou členů týmu, kterou tým předloží družstvu.}

    (\enquote{The entry of individual persons into the team and their departure from it is governed by an internal agreement between team members, which is submitted to the cooperative.})
\end{quote}

A team is represented by a single statutory representative and must submit its internal agreement to the cooperative whenever its member structure changes.

%- - - - - - - - - - - - - - - - - - - - - - - - - - - - - - - -
\paragraph{Capital Structure}
Each member must pay a basic contribution of 500 CZK to join. Additional contributions can be made voluntarily or required by member vote under specific conditions (e.g., to increase capital every 3 years as stated in § 566 of the Act on Commercial Corporations).
\begin{quote}
    Article I, Section 5: \czquote{Výše základního členského vkladu [...] činí 500,- Kč. Člen se [...] může podílet na základním kapitálu družstva současně i jedním nebo více dalšími členskými vklady.}

    (\enquote{The amount of the basic membership contribution is CZK 500. A~member may also contribute to the cooperative's share capital with one or~more additional membership contributions.})
\end{quote}
\begin{quote}
    Article XII, Section 1: \czquote{Členská schůze (a.) rozhoduje o využití fondu na následující čtvrtletí}

    (\enquote{The members’ assembly (a.) decides on the use of the fund for the~upcoming quarter.})
\end{quote}
\begin{quote}
    Article XIV, Section 4: \czquote{Předseda odpovídá za operativní nakládání s~fondem v souladu s usnesením členské schůze.}

    (\enquote{The chairman is responsible for the day-to-day management of the~fund in accordance with the resolution of the members’ meeting.})
\end{quote}

Contributions are non-refundable during membership, except when legally reduced. A fund is established to uphold the operations of the cooperative. The~members’ assembly directs the fund’s usage; the chairman is responsible for day-to-day operational management, within the limits of that direction.

%- - - - - - - - - - - - - - - - - - - - - - - - - - - - - - - -
\paragraph{Governance Structure}
\begin{itemize}
    \item \textbf{Members’ Assembly (Členská schůze):} Highest authority, includes all members.
    \item \textbf{Chairman (Předseda):} Elected statutory body; 5-year term.
\end{itemize}
Members can vote, be elected, and participate in decisions, including online if the platform allows identity verification. Once the organisation exceeds 49~members the statues need to be redrafted and a board of directors formed.

Each member (individual, team, or legal entity) has one vote. Proxy representation is limited to one third of members.\footnote{A member can represent (power of attorney) one third of the present votes at maximum.}
%- - - - - - - - - - - - - - - - - - - - - - - - - - - - - - - -
\paragraph{Profit Sharing}
Profits are distributed quarterly if the cooperative is solvent, with 5\% set aside for a cooperative fund.

Each member receives a share proportionate to their contributions (profits generated) in the accounting period. Teams split their share equally among members.
\begin{quote}
    Article VI, Section 1: \czquote{Po odečtení části 5\%, která je určena k tvorbě fondu družstva, má člen právo na podíl výdělku odpovídající výdělku z jeho činnosti...}

    (\enquote{After deducting 5\% for the cooperative’s fund, a member is entitled to a share of the earnings corresponding to the income generated by their own activities [...]})
\end{quote}

%- - - - - - - - - - - - - - - - - - - - - - - - - - - - - - - -
\paragraph{Rights and Responsibilities}
Members have the right to:
\begin{itemize}
    \item vote and be elected,
    \item receive profit and liquidation share,
    \item access cooperative information and internal records.
\end{itemize}
Members are obligated to:
\begin{itemize}
    \item follow the statutes and terms of service,
    \item respect valid decisions of the cooperative’s bodies.
\end{itemize}

%- - - - - - - - - - - - - - - - - - - - - - - - - - - - - - - -
\paragraph{Exit and Expulsion}
Membership (in accordance with § 610 of the Act on~Commercial Corporations) ends via:
\begin{itemize}
    \item voluntary exit (written notice),
    \item mutual agreement,
    \item death or dissolution of a legal entity,
    \item expulsion (for serious or repeated violations, with formal warning and opportunity to rectify).
\end{itemize}

Statutory safeguards ensure due process, including:
\begin{itemize}
    \item decision must have written form,
    \item notice period (minimum 30 days),
    \item right to appeal to the members’ assembly and a court.
\end{itemize}

%- - - - - - - - - - - - - - - - - - - - - - - - - - - - - - - -
\paragraph{Digital Infrastructure}
Official communication and records (e.g., noticeboard, meetings) can be maintained via an electronic channel, which must verify identity and ensure timely delivery of documents (e.g., voting results).
\begin{quote}
    Article I, Section 10: \czquote{Družstvo zřídilo ve svém sídle a ve svém oficiálním komunikačním kanálu informační desku […]}

    (\enquote{The cooperative has established an information board at its registered office and on its official communication channel […]})
\end{quote}
\begin{quote}
    Article IX, Section 2: \czquote{Členská schůze se může konat prostřednictvím oficiálního elektronického komunikačního kanálu, který umožňuje ověření totožnosti členů.}

    (\enquote{The members’ assembly may be held via the official electronic communication channel, which enables verification of members’ identities.})
\end{quote}

These decisions were made to ensure the cooperative remains accessible, sustainable, and is realistically aligned with the traits of a student-led organisation. Allocating the initial capital to partially cover administrative costs---such as registration and accounting---helps reduce the financial burden during the~early stages when revenue may be limited. In a traditional cooperative, revenue is distributed based on the capital devoted by a member. I defined teams, with equal rights and votes regardless of additional capital provided. Additionally, the statutes dictate that members’ assemblies take place each quarter, to approve the distribution of profits  based on the performance of a~team’s game. Allowing meetings and decision-making processes to take place online ensures participation from all members, advancing transparency and inclusivity. Terms of service can shortly define prohibited behaviour to shield members and the incubator from liability. This flexible, low-overhead structure fits the cooperative's core needs: equipping young creators while minimizing bureaucracy. The statutes can be found in their entirety in appendix \ref{ap:statutes}.

%===============================================================
\section{Setting Up the Cooperative}\label{sec:coop-setup}
A cooperative in the Czech Republic can be established without a founding meeting, by executing a notarial founding act, as permitted by Section 561a of Act No. 90/2012 Coll., on Commercial Corporations\cite{ZOK}. This procedure:
\begin{itemize}
    \item is faster and simpler---useful when there are fewer founders,
    \item allows for more efficient communication among founders, and
    \item unifies documentation in one notarial act.
\end{itemize}
A cooperative must be founded by at least three persons. Founders may be either individuals or legal entities and must be present at the notarial act. The~notarial deed includes:
\begin{itemize}
    \item the cooperative’s name and registered office
    \item Its scope of business or activity
    \item the amount and form of basic membership contributions
    \item the identity and appointment of the initial members of governing bodies (e.g., chairman or board of directors, supervisory board if applicable)
    \item the cooperative's articles (statutes)---included in appendix \ref{ap:statutes} 
    \item a declaration by the founders that they meet all legal conditions for membership.\cite{coop-funding}
\end{itemize}
Each founder must fulfil their initial capital contribution prior to submitting the registration application to the Commercial Register. The contributions can be paid either to a designated account or to a trustee (e.g., one of the~founders or a notary). There is no required minimum capital for cooperatives, but all contributions must be clearly documented.
\cite{coop-funding}

After signing the founding deed and fulfilling all capital contribution obligations, the cooperative must be registered in the Commercial Register. The~application is submitted by the chairman or another authorized founder to the~regional commercial court and includes:
\begin{itemize}
    \item the notarial deed of incorporation (founding deed),
    \item the articles of association,
    \item proof of capital contributions,
    \item consent to the seat of the cooperative (if not owned by the cooperative),
    \item declarations and criminal record extracts of the governing body members,
    \item other required documentation based on the nature of the cooperative. \cite{coop-funding}
\end{itemize}
The typical costs associated with this method of establishment include:
\begin{itemize}
    \item \textbf{Notarial deed:} Approximately CZK~5~000 - CZK~10~000, depending on~the complexity.
    \item \textbf{Court administered registration into the commercial registry fee:} CZK~2~000 (for electronic filings).
    \item \textbf{Additional administrative costs:} Around CZK~500 - CZK~1~000 (e.g., document copies, bank statement, criminal records, signature verifications). \cite{coop-funding}
\end{itemize}

%===============================================================
\section{Staffing and Funding the Cooperative}
The operations involve navigating a range of administrative responsibilities that go far beyond creative support and mentoring.

Inspired by best practices from UK and US university incubators\cite{isis-innovation}, Grafit.games should be staffed leanly, aiming to support 10–30 active teams per year:
\begin{itemize}
    \item \textbf{Governance:} administrative directing and oversight should be carried out by the incubator chairman (in accordance with section \ref{sec:defining-coop}).
    \item \textbf{Core Team:} 1--2 employees with creative industry, game development, or~entrepreneurial experience.
    \item \textbf{Support Staff:} Student workers or recent alumni providing assistance with communications, testing, mentoring, or event management.
    \item \textbf{Advisory Panel (If Possible):} Composed of university faculty, industry mentors, and cooperative members.
\end{itemize}
The incubator aims to connect (possibly through the advisory panel) to faculty research departments, university entrepreneurship centres, or the tech transfer office.

To get Grafit.games off the ground, initial funding is required\cite{strategy-incubation}. It needs to cover the setup (section \ref{sec:coop-setup}) and operational cost during the pilot run---described in chapter \ref{chap:launch-testing}. In the later stages, revenue is hoped to uphold a larger portion of operational costs. Funding may be sourced from:
\begin{itemize}
    \item university or faculty seed funding,
    \item grant programmes (EU regional funds, Czech regional or innovation funds),
    \item sponsorship from industry partner,
    \item cooperative membership fees,
    \item revenue share from successful projects.
\end{itemize}

%===============================================================
\section{Admitting Teams to the Cooperative}
A core philosophy of the incubator is a founder-friendly and flexible approach to intellectual property (IP). I, therefore, propose a hybrid IP model:
\begin{itemize}
    \item Students always own the IP – both moral (to be listed as the author) and economic rights (to direct the commercial use) remain with the creators.
    \item The cooperative receives a non-exclusive commercial licence to:
    \begin{itemize}
        \item publish, promote, distribute, and sell the game;
        \item attain a modest share of net revenue (5\%).
    \end{itemize}
\end{itemize}
If a team leaves the cooperative, they may terminate the licence and retain all rights to the game. Alternatively, if it benefits both sides, the licence can remain in effect.

As mentioned in subsection \ref{subsec:legal-requirements}, work created under an employment contract automatically surrenders economic rights to the employer; however, this section does not apply in the case of membership in a cooperative. The students, therefore, retain all rights and the incubator requires a licence to be allowed to sell a game.

In accordance with the Copyright Act and Dean’s Directive No. 37/2019 at~FIT~CTU, the school has a right to license the work that is created:
\begin{itemize}
    \item as part of coursework or a thesis,
    \item using university infrastructure such as labs, or
    \item through direct involvement of university staff. \cite{Kurzy_autorsky-zakon, FIT-smernice}
\end{itemize}
To avoid legal fallout with the university, students attempting to commercialise games created in one of the mentioned ways will be required to contact the~dean, state the Grafit.games incubator as the legal entity that will carry out the commercialisation in accordance with section 5 of the directive, and initiate an IP licensing agreement in accordance with section 3. This agreement should clearly outline:
\begin{itemize}
    \item licensing terms including:
    \begin{itemize}
        \item exclusivity of the licence,
        \item showcase of the game for promotional purposes,
        \item use in the educational process,
        \item geographic constraints,
        \item time constraints.
    \end{itemize}
    \item scope of commercialisation rights granted to Grafit.games,
    \item revenue-sharing arrangements (e.g., percentage of net profit) between the~author, university, and the publisher---Grafit.games. \cite{FIT-smernice}
\end{itemize}
