\externaldocument[nocite]{build/text/chapters/support-formats}[url=text/chapters/support-formats]
\externaldocument[nocite]{build/text/chapters/launch-&-testing}[url=text/chapters/launch-&-testing]
\externaldocument[nocite]{build/text/appendix}[url=text/appendix]
%===============================================================
\chapter{Legal Structure for Our Mechanism}
%===============================================================

\begin{chapterabstract}	
    Student-led game development is inherently creative and unpredictable, often lacking the business structure needed to succeed in a competitive market. Our incubators can provide that structure. This chapter evaluates various legal models for university-affiliated incubators, ultimately proposing a phased approach beginning with informal student clubs and transitioning into a cooperative structure. The cooperative model has been chosen for its democratic governance, shared ownership, ability to fairly distribute profits among members, and low bureaucracy. It incorporates informal teams, giving them a legal structure. The chapter provides statutes - describing membership, governance, and capital contributions - and guides creation of the entity. Additionally it describes the handling of Intellectual property rights using non-exclusive licenses - allowing the incubator to commercialize games and student developers to retain rights to their work.
\end{chapterabstract}

Transitioning a prototype to a market-ready product is an exciting yet formidable journey. Providing mentorship and consulting goes a long way in facilitating that mission. Game development ventures, particularly at the student level, are however unpredictable by nature - they rely on creative momentum in a competitive market. As such, navigating the legal, financial, and administrative landscape can pose an insurmountable obstacle. It led us to conclude in [CHAPTER SWOT] that business support should be a prominent feature of our programme. While such services can begin informally, with growing complexity establishing a formal legal structure becomes necessary. It allows for consistency, effective management of responsibilities, and ensures long-term sustainability. The selection of an appropriate legal structure is therefore critical.

%===============================================================
\section{Selecting a Legal Structure}
In chapter~\ref{chap:support-formats}, we outlined the various models for supporting student ventures and the legal forms university incubators may take in the Czech Republic. Each offers distinct benefits and constraints.

%---------------------------------------------------------------
\subsection{Forms of Support}
While financial support is a valuable feature of a support programme, evidence from other university incubators suggests that equity- or royalty-based models are rarely sustainable. As such, we intend to offer monetary support primarily in the later stages of a project and, wherever possible, free of charge.

Directly employing students is not aligned with our objectives. Czech labor laws require that students be financially compensated for work performed—unless they hold ownership in a company. Without a sustainable revenue model, such a framework becomes infeasible.

Instead, our focus will be on offering mechanisms for student ownership within the incubator. This approach allows us to centralize administrative and legal responsibilities, reducing the burden on student developers. By incorporating student games under a unified structure, we can manage accounting, legal compliance, and profit distribution on their behalf.

%---------------------------------------------------------------
\subsection{Incubator Integrated into the Non-Profit University}
Three primary models exist for integrating an incubator within the university’s non-profit structure: Student Clubs, Public-Private Partnerships, and Non-Profit Organizations.

\textbf{Student Clubs} are informal, low-cost collectives of students and staff. They provide a mechanism useful for testing the educational model. They however lack the legal and commercial tools necessary for supporting ventures beyond an academic context. Business support would remain entirely the students’ responsibility.

\textbf{Public-Private Partnerships} could involve university educators mentoring students while private studios handle publishing. While valuable for high-potential projects, these collaborations are often difficult to secure and benefit only a small subset of participants. They are unlikely to offer broad commercial support for most student ventures.

\textbf{Non-Profit Organizations} benefit from grant eligibility and tax exemptions but are subject to intensive regulation, requiring complex administration and robust bookkeeping. They cannot directly distribute profits to students. While not ideal for the early commercial phases, a non-profit may become a suitable option once the educational mechanism is validated and sufficient funding and partners are secured.

%---------------------------------------------------------------
\subsection{Incubator as a Spin-Off Company}
Creating a separate legal entity offers the benefit of autonomy from university bureaucracy. Several models offer different levels of profit distribution, flexibility and scalability:

\textbf{A joint-stock company} is well-suited for attracting investors, scaling large ventures, and provides strong mechanisms for profit distribution. However, it requires at least CZK 2 million in initial capital. It is more appropriate for the later stages of the incubator and large-scale initiatives.

\textbf{A limited liability company} has low startup costs and a straightforward method for distributing profits via dividends to owners—avoiding the need for social or health insurance contributions. It might be ideal for student-founded start-ups, but since ownership changes require recording in the commercial register (CZK 2000 fee), it does not easily allow us to incorporate students. This model is therefore solely an extension of the student club structure.

\textbf{A cooperative} allows students, faculty, and partners to share ownership, profits, and decision-making. Though administratively more complex than an s.r.o., it offers a model well-suited to our long-term goals.

%---------------------------------------------------------------
\subsection{Our Priorities}
To balance simplicity and potential, we have selected a phased approach:

\begin{enumerate}
    \item \textbf{Initial Phase} – we will begin with an unofficial student club to pilot the educational portion of the programme. This low-cost model allows for rapid testing and adaptation without formal legal obligations.
    \item \textbf{Growth Phase} – once demand for business support emerges, we will formalise operations using one of two structures:
    \begin{itemize}
        \item \textbf{An LLC (s.r.o.)}, ideal for handling commercial transactions, investing in student projects, or acquiring equity in student-founded companies.
        \item \textbf{A Cooperative}, for collaborative ventures where shared ownership and reduced administrative burden are prioritized.
    \end{itemize}
    \item \textbf{Future Phase} – if the incubator attracts significant partners, investors, or establishes a large fund, it can be transitioned into a Joint-Stock Company, Fund, or Foundation, depending on its evolved purpose and scale.
\end{enumerate}

This step-by-step approach ensures that our mechanism adapts and is legally adequate.


%===============================================================
\section{Defining the Cooperative}
Setting up a cooperative allows democratic governance, equitable profit sharing, and joint control of creative and financial direction. By implementing the laid out structure a collective of developers can formalize collaboration without sacrificing autonomy or flexibility. The following chapter outlines key characteristics of statutes found in appendix \ref{ap:statutes}. They were created based on similar existing documents.[ADD REFERENCES]
%- - - - - - - - - - - - - - - - - - - - - - - - - - - - - - - -
\paragraph{Legal Form and Purpose}
The cooperative must be established as a production cooperative (výrobní družstvo), a type of business corporation under the Czech Act on Business Corporations (§ 552– ZOK, zákon č. 90/2012 Sb.). It is open-ended in terms of number of members and serves both a business purpose and the common interests of its members.

\begin{quote}
    Article I, Section 4: \czquote{Družstvo je společenstvím neuzavřeného počtu osob založeným za účelem podnikání. Družstvo je obchodní korporací podle zákona o obchodních korporacích.}
\end{quote}

The declared scope of activity should include:
\begin{itemize}
    \item Game development (including related services like QA, graphics, music, writing, publishing, events),
    \item Marketing and media services in digital entertainment,
    \item R\&D in digital technologies and media.
\end{itemize}
%- - - - - - - - - - - - - - - - - - - - - - - - - - - - - - - -
\paragraph{Membership}
Membership is open to:
\begin{itemize}
    \item Individuals over 18 years old,
    \item Teams (at least 2 individuals with an internal agreement),
    \item Legal entities (companies, organizations).
\end{itemize}
An applicant must submit a simple written application. Membership is granted by decision of the cooperative’s chairman. Application must include:
\begin{itemize}
    \item Name and address (for individuals),
    \item Team name and addresses of members (for teams of individuals),
    \item Legal name and registered office (for legal bodies),
    \item Address (digital) for official communication.
\end{itemize}

\begin{quote}
    Article III, Section 4: \czquote{Tým je v družstvu zastoupen jedním statutárním zástupcem, který vykonává práva a povinnosti člena družstva za celý tým.}
\end{quote}

\begin{quote}
    Article III, Section 5: \czquote{Vstup jednotlivých fyzických osob do týmu a jejich vystoupení z týmu se řídí vnitřní dohodou členů týmu, kterou tým předloží družstvu.}
\end{quote}

A team is represented by a single statutory representative and must submit its internal agreement to the cooperative whenever their member structure changes.
%- - - - - - - - - - - - - - - - - - - - - - - - - - - - - - - -
\paragraph{Capital Structure}
Each member must pay a basic contribution of 500 CZK to join. Additional contributions can be made voluntarily or required by member vote under specific conditions (e.g., to increase capital every 3 years, capped at a 3x increase). 

Key Legal Reference: § 563 ZOK – Member contributions and increase rules.
\begin{quote}
    Article I, Section 5: \czquote{Výše základního členského vkladu [...] činí 500,- Kč. Člen se [...] může podílet na základním kapitálu družstva současně i jedním nebo více dalšími členskými vklady.}
\end{quote}
\begin{quote}
    Article XII, Section 1: \czquote{Členská schůze (a.) rozhoduje o využití fondu na následující čtvrtletí}
\end{quote}
\begin{quote}
    Article XIV, Section 4: \czquote{Předseda odpovídá za operativní nakládání s fondem v souladu s usnesením členské schůze.}
\end{quote}

Contributions are non-refundable during membership, except when legally reduced. A fund is established to uphold the operations of the cooperative. The members’ assembly directs the fund’s usage, the chairman is responsible for day-to-day operational management, within the limits of that direction.
%- - - - - - - - - - - - - - - - - - - - - - - - - - - - - - - -
\paragraph{Governance Structure}
\begin{itemize}
    \item \textbf{Members’ Assembly (Členská schůze)} - Highest authority, includes all members.
    \item \textbf{Chairman (Předseda)} - Elected statutory body; 5-year term.
\end{itemize}
Members can vote, be elected, and participate in decisions, including online if the platform allows identity verification. Once the organisation exceeds 49 members the statues need to be redrafted and a board of directors formed.

Each member (individual, team, or legal entity) has one vote. Proxy representation is limited to one third of members.\footnote{A member can represent (power of attorney) one third of the present votes at maximum.}
%- - - - - - - - - - - - - - - - - - - - - - - - - - - - - - - -
\paragraph{Profit Sharing}
Profits are distributed quarterly if the cooperative is solvent, with 5\% set aside for a cooperative fund.

Each member receives a share proportionate to their contributions (profits generated) in the accounting period. Teams split their share equally among members.
\begin{quote}
    Article VI, Section 1: \czquote{Po odečtení části 5\%, která je určena k tvorbě fondu družstva, má člen právo na podíl výdělku odpovídající výdělku z jeho činnosti...}
\end{quote}

%- - - - - - - - - - - - - - - - - - - - - - - - - - - - - - - -
\paragraph{Rights and Responsibilities}
Members have the right to:
\begin{itemize}
    \item Vote and be elected,
    \item Receive profit and liquidation share,
    \item Access cooperative information and internal records.
\end{itemize}
Members are obligated to:
\begin{itemize}
    \item Follow the statutes and terms of service,
    \item Respect valid decisions of the cooperative’s bodies.
\end{itemize}
%- - - - - - - - - - - - - - - - - - - - - - - - - - - - - - - -
\paragraph{Exit and Expulsion}
Membership ends via:
\begin{itemize}
    \item voluntary exit (written notice)
    \item mutual agreement
    \item death or dissolution of a legal entity
    \item expulsion (for serious or repeated violations, with formal warning and opportunity to rectify)
\end{itemize}
Key Legal Reference: § 610 ZOK – Member expulsion procedure.

Statutory safeguards ensure due process, including:
\begin{itemize}
    \item decision must have written form
    \item notice period (minimum 30 days)
    \item right to appeal to the members’ assembly and a court
\end{itemize}
%- - - - - - - - - - - - - - - - - - - - - - - - - - - - - - - -
\paragraph{Digital Infrastructure}
Official communication and records (e.g., noticeboard, meetings) can be maintained via an electronic channel, which must verify identity and ensure timely delivery of documents (e.g., voting results).
\begin{quote}
    Article I, Section 10: \czquote{Družstvo zřídilo ve svém sídle a ve svém oficiálním komunikačním kanálu informační desku…}
\end{quote}

\begin{quote}
    Article IX, Section 2: \czquote{Členská schůze se může konat prostřednictvím oficiálního elektronického komunikačního kanálu, který umožňuje ověření totožnosti členů.}
\end{quote}

These decisions were made to ensure the cooperative remains accessible, sustainable, and is realistically aligned with the traits of a student-led organization. Allocating the initial capital to partially cover administrative costs - such as registration and accounting - helps reduce the financial burden during the early stages when revenue may be limited. Revenue is in a traditional cooperative distributed based on the capital devoted by a member. We define teams, with equal rights and votes regardless of additional capital provided. Additionally we dictate members’ assemblies to take place each quarter to approve the distribution of profits  based on the performance of a team’s game. Allowing meetings and decision-making processes to take place online ensures participation from all members, advancing transparency and inclusivity. Terms of service can shortly define prohibited behaviour to shield members and the incubator from liability. This flexible, low-overhead structure fits the cooperative's core needs: equipping young creators while minimizing bureaucracy. The statutes can be found in appendix \ref{ap:statutes}.

%===============================================================
\section{Setting Up the Cooperative}
A cooperative in the Czech Republic can be established without a founding meeting, by executing a notarial founding deed, as permitted by Section 561a of Act No. 90/2012 Coll., on Commercial Corporations. This procedure:
\begin{itemize}
    \item Is faster and simpler - useful when there are fewer founders.
    \item Allows for more efficient communication among founders.
    \item Unifies documentation in one notarial act.
\end{itemize}
A cooperative must be founded by at least three persons. Founders may be either individuals or legal entities. All founders must be present at the notarial act.

The cooperative is established through a notarial deed that includes:
\begin{itemize}
    \item The cooperative’s name and registered office.
    \item Its scope of business or activity.
    \item The amount and form of basic membership contributions.
    \item The identity and appointment of the initial members of governing bodies (e.g., chairman or board of directors, supervisory board if applicable).
    \item The cooperative's articles (statutes), attached as an annex.
    \item A declaration by the founders that they meet all legal conditions for membership.
\end{itemize}
An article of association (statutes) must be agreed upon by all founders and attached to the founding deed. It must define:
\begin{itemize}
    \item Membership rights and obligations.
    \item Organizational structure and governing bodies.
    \item Rules for general meetings.
    \item Voting procedures.
    \item Capital and financial management.
    \item Profit distribution and loss coverage.
    \item Termination and transformation procedures, among others.
\end{itemize}
Each founder must fulfill their initial capital contribution prior to submitting the registration application to the Commercial Register. The contributions can be paid either to a designated account or to a trustee (e.g., one of the founders or a notary). There is no required minimum capital for cooperatives, but all contributions must be clearly documented.

After signing the founding deed and fulfilling all capital contribution obligations, the cooperative must be registered in the Commercial Register. The application is submitted by the chairman or another authorized founder to the regional commercial court and includes:
\begin{itemize}
    \item The notarial deed of incorporation (founding deed).
    \item The articles of association.
    \item Proof of capital contributions.
    \item Consent to the seat of the cooperative (if not owned by the cooperative).
    \item Declarations and criminal record extracts of the governing body members.
    \item Other required documentation based on the nature of the cooperative.
\end{itemize}
The typical costs associated with this method of establishment include:
\begin{itemize}
    \item \textbf{Notarial deed:} approximately CZK~5~000 - CZK~10~000, depending on the complexity.
    \item \textbf{Court administered registration into the commercial registry fee:} CZK~2~000 (for electronic filings).
    \item \textbf{Additional administrative costs:} around CZK~500 - CZK~1~000 (e.g., document copies, bank statement, criminal records, signature verifications).
\end{itemize}

%===============================================================
\section{Funding the Cooperative}
To get Grafit.games off the ground, initial funding is required. It needs to cover the setup, initial operational cost [MENTIONED IN 6.4], and at least a pilot run described in chapter \ref{chap:launch-testing}. In the later stages, revenue is hoped to uphold a larger portion of operational costs. Funding may be sourced from:
\begin{itemize}
    \item university or faculty seed funding
    \item grant programmes (EU regional funds, Czech regional or innovation funds)
    \item sponsorship from industry partner
    \item cooperative membership fees
    \item revenue shares from successful projects
\end{itemize}

%===============================================================
\section{Admitting Teams to the Cooperative}
A core philosophy of the incubator is a founder-friendly and flexible approach to intellectual property (IP), we therefore propose a hybrid IP model:
\begin{itemize}
    \item Students always own the IP – both moral (to be listed as the author) and economic rights (to direct the commercial use) remain with the creators.
    \item The cooperative receives a non-exclusive commercial license to:
    \begin{itemize}
        \item publish, promote, distribute, and sell the game
        \item attain a modest share of net revenue (5\%)
    \end{itemize}
\end{itemize}
If a team leaves the cooperative they may terminate the license and retain all rights to the game. Alternatively, if it benefits both sides, the license can remain in effect. 

Section 12(1) of the Copyright Act No. 121/2000 Coll. grants an author exclusive right to use and authorise the use of their work. Section 58 states that work created as part of fulfilling employment duties automatically surrenders the economic rights to the employer, however, this section does not apply in the case of membership in a cooperative. The students therefore retain all rights and the incubator requires a license to be allowed to sell a game.

The Copyright Act No. 121/2000 Coll. also outlines certain rights for a university. In accordance with section 35(3) and Dean’s Directive No. 37/2019 at FIT CTU, students retain full ownership of any games or work they develop independently, outside of coursework or institutional resources. No university approval is required for further commercial use in such cases. However, if a student-developed game is created in at least one of the following ways:
\begin{itemize}
    \item as part of coursework or a thesis
    \item using university infrastructure such as labs
    \item through direct involvement of university staff
\end{itemize}
Then whilst the student still owns both the moral and economic rights, according to Section 60 of the Copyright Act, the school has a right to license the work. To avoid legal fallout with the university, the student can contact the dean in accordance with section 5 of the directive, state the Grafti.games incubator as the legal entity that will carry out the commercialisation, and initiate an IP licensing agreement in accordance with section 3. This agreement should clearly outline:
\begin{itemize}
    \item Licensing terms - whether the licence is exclusive, if it allows showcase of the game for promotional purposes and if it allows the game to be used in the educational process. Additionally, it should outline the geographic and time constraints of the license.
    \item The scope of commercialization rights granted to Grafit.games.
    \item Revenue-sharing arrangements (e.g., percentage of net profit) between the author, university, and the publisher - Grafit.games.
\end{itemize}
To summarise, students own the ethical and economical right to work they have created. In the Czech Republic, finalising a game whilst part of a cooperative does not transfer rights, therefore to allow the incubator to sell a game a licensing agreement needs to be devised. If the game has been developed as part of the student’s academic obligations, the faculty has the right to take part in the licensing agreement and will likely aim to get reimbursed for the committed resources.
